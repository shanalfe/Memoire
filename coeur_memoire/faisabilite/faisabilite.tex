\commentaire{\\
    •	Quels obstacles (coût, poids, énergie, connectivité, acceptabilité des motards) à l’implémentation ?\\
	•	Quelles pistes pour la recherche ou le développement industriel ?\\
	•	(une maquette fonctionnelle, un prototype conceptuel, ou même une étude de cas simulée)\\
  •	ouverture défi environnemental\\
  •	 protection des données sensibles (lien via kappa)        }

Les limites actuellement restent les prix. En effet, ajouter des fonctionnalités de sécurité ayant des prix trop important baissent l'attractivité des motos. 


Comme il n’existe pas encore de produit commercial alliant GPS et l'alerte virage, on peut estimer sur la base des coûts de prototypes et de composants.\\

\begin{tabular}{|l|l|}
\hline
\textbf{Élément} & \textbf{Estimation de coût} \\
\hline
Capteur GPS + IMU (accéléro/magnétomètre)  & 50–200 € \\
Abonnement cartes HD (HERE, TomTom, etc.) &    10–30€ par mois   \\
Interface casque ou écran moto (HUD/haptique) au besoin & 100–300€ \\
\hline
\end{tabular}