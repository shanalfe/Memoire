\commentaire{\\
    •	Quels obstacles (coût, poids, énergie, connectivité, acceptabilité des motards) à l’implémentation ?\\
	•	Quelles pistes pour la recherche ou le développement industriel ?\\
	•	(une maquette fonctionnelle, un prototype conceptuel, ou même une étude de cas simulée)\\
  •	ouverture défi environnemental\\
  •	 protection des données sensibles (lien via kappa)        }

\todo{mon programme}
Le programme fonctionne dans la pluspart des cas. Cependant, les données liant la limitation de vitesse n'est pas toujours récupérée. Par conséquent, par défaut la limitation de vitesse est 80 km/h. Sur les portions d'autoroute, la limitation est de 130 km/h et la courbure a une valeur quasiment nulle. Prenons un cas où la limitation est 110 km/h, il faut que la courbure soit suffisamment élevée pour ne pas pouvoir prendre un virage à plus de 80 km/h. C'est un risque accepté et réfléchi pour ce programme.
Concernant la fonctionnalité, certains motards peuvent percevoir cette aide comme trop "prévoyante". En effet, plus un motard roule et plus il acquière de l'expérience, sous condition qu'il varie les types de trajet. Il peut donc se dire que l'alerte arrive trop tôt. On peut donc prendre en concidération une variable de niveau qui propose un déclanchement de l'alerte selon une certaine vitesse. L'accélération est un facteur que l'on peut également prendre en compte. À un niveau de jet avant un virage peut suffir pour avertir d'un danger.

Les limites actuellement restent les prix. En effet, ajouter des fonctionnalités de sécurité ayant des prix trop important baissent l'attractivité des motos. 


Comme il n’existe pas encore de produit commercial alliant GPS et l'alerte virage, on peut estimer sur la base des coûts de prototypes et de composants.\\

\begin{tabular}{|l|l|}
\hline
\textbf{Élément} & \textbf{Estimation de coût} \\
\hline
Capteur GPS + IMU (accéléro/magnétomètre)  & 50–200 € \\
Abonnement cartes HD (HERE, TomTom, etc.) &    10–30€ par mois   \\
Interface casque ou écran moto (HUD/haptique) au besoin & 100–300€ \\
\hline
\end{tabular}

\vspace{0.5cm}
Afin d'avoir un GPS précis et réactif. Il doit être accompané de la technologie IMU. IMU permet de compenser la latence et les imprécisions GPS en fournissant la vitesse angulaire (gyroscope) pour la courbure des virages, l'orientation (magnétomètre) et l'accélération linéaire (accéléromètre) pour les freinage. L'IMU peut fonctionner à 100-1000 Hz ce qui permet d'améliorer fortement le temps de réaction. Ces points sont crutiaux pour une fonctionnalité comme la notre, la prédiction d'abord de virages dangereux. La marge d'erreur n'est pas permise.

Voici un tableau comparatif:\\
\begin{tabular}{|c|c|c|}
\hline
\textbf{Propriétés} & \textbf{GPS} & \textbf{GPS + IMU} \\
\hline
Précision position & Environ 3 à 5 m & Bonne \\
Temps de réaction & Lent (de 0,5 à 1 seconde) & Rapide (<0,1 seconde)\\
Fiabilité en virage & Faible & Haute \\
Sensibilité du signal & Oui & Moins critique \\
\hline
\end{tabular}

\todo{développer la partie loi + internationnal}
L'utilisation du GPS est très réglementé, surtout dans les pays. Le défi ici c'est de se demander comment peut-on développer des technologies, des innovations qui respectent les règles, la confidentialité des usagers des pays concernés.\\

Par exemple :\\
La Chine autorise le GPS mais avec un accès limité à certains services cartographiques. Dans plusieurs pays comme l'Autriche, l'Allemagne n'autorise pas les GPS avec les avertisseurs ou détecteurs de radar. Au Japon, le GPS est autorisé mais certains appareils RF\footnote{Radio Fréquence : Ce sont des dispositifs qui émettent et recçoivent des ondes radio pour communiquer entre 3 kHz et 300 GHz.} sont interdits.


\todo{défi affichage en respectant la loi}
L'utilisation du GPS est autorisé quand celui-ci ne gêne ni la visibilité, ni la conduite d'après Code de la route français – Art. R412-6\cite{loi_code_de_la_route}.

\todo{rgpd}
Le GPS va recueillir des données, suivre l'usager. Pour approfondir et améliorer le système, des données seront collectées par conséquent, le consentement explicite du motard est nécessaire. Il faut également ajouter une déclaration ou une politique de confidentialité claire.

\todo{proposition rédaction rgpd}
Il faudra collecter que les données nécessaires position, vitesse, horodatage. Les données doivent être stockées dans une base de données sécurisées et chifrées.

\todo{choix, marque du produit}
Voici des propositions de composants que nous pouvons utiliser : 	\\
•	u-blox NEO-M8N : excellent rapport qualité/prix, ~40 € \\
•	u-blox ZED-F9P : haute précision RTK, mais cher (~200–250 €) \\
•	SparkFun GPS + IMU : combine ZED-F9P + BNO080 (IMU)\\


