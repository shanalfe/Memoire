\commentaire{\\
    •	Quels obstacles (coût, poids, énergie, connectivité, acceptabilité des motards) à l’implémentation ?\\
	•	Quelles pistes pour la recherche ou le développement industriel ?\\
	•	(une maquette fonctionnelle, un prototype conceptuel, ou même une étude de cas simulée)\\
  •	ouverture défi environnemental\\
  •	 protection des données sensibles (lien via kappa)        }


\subsubsection{Objectif de l'étude}
Cette étude vise à évaluer la faisabilité de la mise en œuvre d’un système d’assistance à la conduite basé sur l’analyse de la courbure de la route, dans le but de recommander une vitesse adaptée. Le système exploite des données géographiques pour déterminer la géométrie des segments routiers en fonction de la position d’un véhicule. Puis, il applique un modèle simple de calcul de courbure pour estimer la dangerosité d’un virage et proposer une vitesse conseillée. Ce chapitre examine les obstacles techniques, les contraintes d’implémentation réelle, ainsi que les perspectives de développement.
De plus, comme évoqué précédemment, cette assistance n’agit pas directement sur le contrôle du véhicule. D’un point de vue légal, cela rend l’analyse des responsabilités plus simple et limite les zones d’ombre en cas d’incident.

\subsubsection{Faisabilité technique}
Le programme que je propose utilise des données routières open source via OpenStreetMap (OSM). Ces données sont accessibles et gratuites. Cependant, on peut mettre en évidence un manque de précision surtout en milieu rural. Certains segments peuvent manquer de points ou contenir des simplifications qui faussent l’évaluation de la courbure.
La méthode mise en œuvre repose sur le calcul géodésique entre trois points consécutifs, p1, p2 et p3 sur un segment de route à partir d'un point GPS, la position actuelle. Cela permet d’obtenir une estimation simple de la courbure.
Cependant, cette approche reste sensible à la densité des points sur les segments (peu de points donc cela implique une mauvaise précision).

\commentaire{Faisabilité d’implémentation sur un véhicule réel}
Le monde du deux-roues posent certaines contraintes matérielles et logicielles. En effet, les capteurs nécessite d’un GPS de très bonne précision, d'un module RTK\footnote{Real-Time Kinematic : Améliore la précision du GPS en temps réel. Utilisé en topographie, drones, agriculture...} pour éviter les erreurs de localisation afin que les calculs soient optimisés. La connectivité doit être parfaite si les cartes ne sont pas embarquées, cela peut poser des limites en zone blanche. Une connexion réseau serait nécessaire. Concernant le matériel embarqué, nous pouvons utiliser un  microcontrôleur\footnote{ (Par exemple : Raspberry Pi, Arduino, ESP32) : Capable d’exécuter les traitements de calcul ou de les transmettre à une plateforme distante.}. La consommation de ces appareils ne doivent pas être trop importante car cela reste un "petit véhicule" avec une petite batterie. Cela pourrait impliquer des pannes comme un alternateur ou une bobine\footnote{Bobine d’allumage : Agit comme un transformateur entre la batterie et les bougies, et sans elle, le moteur ne peut pas démarrer.} HS.
Les limites actuellement restent les prix. En effet, ajouter des fonctionnalités de sécurité ayant des prix trop important baissent l'attractivité des motos. \\
Comme il n’existe pas encore de produit commercial alliant GPS et l'alerte virage, on peut estimer sur la base des coûts de prototypes et de composants.

\begin{table}[h!]
\centering
\begin{tabular}{|p{7cm}|c|}
\hline
\textbf{Élément} & \textbf{Estimation de coût} \\
\hline
Capteur GPS + IMU (accéléromètre / magnétomètre) & 50–200 € \\
Abonnement cartes HD (HERE, TomTom, etc.) & 10–30 € / mois \\
Interface casque ou écran moto (HUD / retour haptique) & 100–300 € \\
\hline
\end{tabular}
\caption{Estimation des coûts des principaux composants d'un système d’assistance moto.}
\label{tab:couts-composants}
\end{table}
Afin d'avoir un GPS précis et réactif, il doit être accompagné de la technologie IMU. IMU permet de compenser la latence et les imprécisions GPS en fournissant la vitesse angulaire (gyroscope) pour la courbure des virages, l'orientation (magnétomètre) et l'accélération linéaire (accéléromètre) pour les freinages. L'IMU peut fonctionner à 100-1000 Hz ce qui permet d'améliorer fortement le temps de réaction. Ces points sont cruciaux pour une fonctionnalité de prédiction d'abord de virages dangereux. La marge d'erreur n'est pas permise dans le domaine de la sécurité d'un usager.

\vspace{0.5cm}
Voici un tableau comparatif:
\begin{table}[h!]
\centering
\begin{tabular}{|p{4.5cm}|c|c|}
\hline
\textbf{Propriété} & \textbf{GPS} & \textbf{GPS + IMU} \\
\hline
Précision de position & Environ 3 à 5 m & Bonne \\
Temps de réaction & Lent (de 0,5 à 1 seconde) & Rapide (<0,1 seconde) \\
Fiabilité en virage & Faible & Haute \\
Sensibilité au signal & Oui & Moins critique \\
\hline
\end{tabular}
\caption{Comparaison des performances entre un GPS seul et un système GPS couplé à une centrale inertielle (IMU).}
\label{tab:gps-vs-imu}
\end{table}

L’acceptabilité d’un tel système dépend également du profil du conducteur. Un motard préfèrera un système non intrusif (affichage sur smartphone, retour haptique) et intuitif. Le système doit être discret afin de ne pas distraire l’attention ni le surcharger d’information. Un bip, un logo pourraient constituer une solution ergonomique.


\commentaire{Prototype et simulation}
Le prototype logiciel développé offre plusieurs fonctionnalités clés :
\begin{itemize}
  \item La position actuelle du véhicule,
  \item Le chargement dynamique d’un réseau routier à partir d’un point GPS,
  \item L’identification du segment de route le plus proche avec une visualisation de l’environnement routier,
  \item Le calcul de la courbure de ce segment,
  \item La recommandation d’une vitesse adaptée à cette courbure,
\end{itemize}
Ce prototype constitue une première base fonctionnelle, ouvrant la voie à de futures expérimentations sur le terrain.

Les évolutions qui peuvent être envisagées pour les prochaines étapes du projet sont les suivantes :
\begin{itemize}
  \item L’intégration du système dans une plateforme embarquée,
  \item L’ajout d’un suivi GPS en temps réel,
  \item Le développement d’une interface utilisateur, visuelle ou auditive,
  \item La réalisation de simulations ou d’expérimentations plus poussées sur circuit fermé
\end{itemize}

\subsubsection{Confidentialité et Données}
\commentaire{Cadre juridique international autour du GPS}
Le développement de technologies d’assistance à la conduite basées sur la géolocalisation, comme celle proposée dans ce projet, soulève d’importants enjeux juridiques, notamment en matière de souveraineté, d’accessibilité aux données cartographiques, et d’usage du GPS selon les pays.
Dans certains États, l’accès aux données cartographiques est limité. Par exemple, la Chine impose l’utilisation de fournisseurs cartographiques agréés par l’État, avec des restrictions sur les données affichées ou exportables. En Autriche et en Allemagne, les dispositifs de géolocalisation sont autorisés mais l’usage de détecteurs ou d'avertisseurs de radars est strictement interdit, y compris lorsqu'ils sont intégrés à un GPS. Le Japon autorise l’usage du GPS mais certains dispositifs RF\footnote{Radio Fréquence : Ce sont des dispositifs qui émettent et reçoivent des ondes radio pour communiquer entre 3 kHz et 300 GHz.}, notamment ceux qui interfèrent avec d'autres systèmes, sont interdits.

Ces exemples illustrent la diversité des contraintes réglementaires à l’échelle internationale, ce qui constitue un défi pour tout projet technologique visant un déploiement global.
\vspace{0.5cm}

\commentaire{Conformité avec le Code de la route français}
En France, l'utilisation d’un GPS est autorisée dans un véhicule, y compris à moto, tant qu’il ne compromet ni la visibilité, ni la sécurité du conducteur. Le Code de la route, Article R412-6\cite{loi_code_de_la_route} précise que "tout conducteur doit se tenir constamment en état et en position d'exécuter commodément et sans délai toutes les manœuvres qui lui incombent".
Ainsi, tout système d’affichage (écran, casque connecté, ...) devra être conçu de manière à ne pas distraire le conducteur ou gêner sa conduite. Ce point est d’autant plus crucial sur les deux-roues motorisés, où la perte d’attention ou une mauvaise visibilité peut avoir des conséquences graves.
\vspace{0.5cm}

\commentaire{Protection des données personnelles (RGPD)}
L'utilisation d'un système GPS implique nécessairement la collecte de données personnelles sensibles, notamment :
\begin{itemize}
\item La position géographique (latitude et longitude),
\item La vitesse instantanée,
\item L'horodatage précis de chaque relevé (timestamp).
\end{itemize}
Ces informations peuvent permettre d'identifier les comportements individuels ou les trajets récurrents d'un utilisateur. Par conséquent, tout dispositif embarqué ou toute application qui traite ces données doit impérativement respecter le Règlement Général sur la Protection des Données (RGPD), en vigueur au sein de l'Union européenne.\\
Afin d'approfondir et d'améliorer le fonctionnement du système, il sera nécessaire de collecter, de suivre et d'analyser ces données GPS. Cependant, avant toute collecte, il convient d'obtenir le consentement explicite et éclairé du motard. Ce consentement doit être associé à une déclaration ou une politique de confidentialité transparente, expliquant clairement la finalité de la collecte, l'utilisation prévue et les modalités de stockage des informations.

\vspace{0.5cm}
\underline{Exemple d’organisation : Expérience à Kappa Santé}\\
Lors de mon alternance chez Kappa Santé, j'ai pu observer un exemple pertinent d'organisation pour le traitement rigoureux des données sensibles. Chaque collaborateur dispose d'un accès strictement limité aux données nécessaires à ses activités. Par exemple, les développeurs utilisent exclusivement des données fictives, évitant ainsi tout risque d'exposition accidentelle ou de manipulation involontaire de données confidentielles.
Les données réelles, destinées aux analyses approfondies et aux études spécifiques, sont uniquement manipulées par des collaborateurs dûment habilités. Cette organisation repose sur l'utilisation de plusieurs bases de données distinctes, dont l'accès est soigneusement régulé selon les rôles et les droits attribués à chaque utilisateur.\\
Ce type d'organisation présente plusieurs avantages majeurs, notamment la discrétion, la confidentialité accrue des informations, ainsi que la réduction significative des risques en matière de sécurité des données. Un tel modèle pourrait être envisagé comme une référence, en particulier dans d'autres contextes sensibles tels que la mobilité connectée ou la sécurité routière. En garantissant à la fois l'efficacité et la protection des informations personnelles, ce modèle renforce la confiance des utilisateurs et assure une conformité stricte aux réglementations en vigueur.\\

Principes clés à respecter : Dans le cadre du traitement de données personnelles, voici les principes fondamentaux à appliquer :
\begin{itemize}
\item Minimisation des données : collecter uniquement les données strictement nécessaires à la réalisation des objectifs fixés.
\item Consentement éclairé : l'utilisateur (ici, le motard) doit être clairement informé de l'objectif précis de la collecte et fournir explicitement son accord.
\item Sécurité du stockage : assurer le stockage des données dans une base sécurisée, idéalement chiffrée, et limiter la durée de conservation au strict minimum requis.
\item Transparence : accompagner tout dispositif, service ou prototype d'une politique de confidentialité accessible et compréhensible par l'utilisateur.
\end{itemize}

Exemples de données pouvant être conservées conformément au RGPD : Voici des exemples de données pertinentes :
\begin{itemize}
\item Latitude et longitude,
\item Vitesse instantanée,
\item Horodatage précis,
\item Identifiant utilisateur pseudonymisé (non directement identifiable).
\end{itemize}

Ces pratiques garantissent le respect de la vie privée des usagers tout en permettant d’exploiter les données efficacement dans le cadre de recherches ou d’améliorations technologiques.
\vspace{0.5cm}

\commentaire{Contraintes techniques liées à la performance et à la sécurité}
Outre les aspects juridiques, la spécificité des motos implique des contraintes techniques supplémentaires. En effet, les deux-roues sont capables d’accélérations rapides et atteignent des vitesses élevées. Le système embarqué doit donc :
\begin{itemize}
  \item Être capable de traiter et d’analyser les données en quasi temps réel,
  \item Offrir une précision suffisante des points GPS (idéalement < 2 m),
  \item Fournir des recommandations sans latence significative,
  \item Être résilient aux pertes de signal GPS, notamment dans les tunnels ou zones urbaines denses.
\end{itemize}
Ces exigences imposent le choix d’un matériel performant, fiable, et optimisé pour des usages embarqués dans des conditions parfois extrêmes (vibrations, chaleur, humidité).
\vspace{0.5cm}

Voici des propositions de composants que nous pourrions utiliser : 
\begin{itemize}
  \item u-blox NEO-M8N : excellent rapport qualité/prix, ~ 40 €
  \item u-blox ZED-F9P : haute précision RTK\footnote{Real Time Kinematic: obtention d'une précision de l’ordre du centimètre.} mais cher (~ 200 – 250 €)
  \item SparkFun GPS + IMU : combine ZED-F9P + BNO080 (IMU) : 450 €
\end{itemize}
\vspace{0.5cm}

\subsubsection{Limites et contraintes}
\commentaire{programme}
Lorsqu’il s’agit d’évaluer la dangerosité d’un virage, la courbure n’est qu’un élément parmi d’autres. L’état de la chaussée, les conditions météo, la visibilité, la signalisation, la présence d'un passager craintif ou non, ou encore la présence d’obstacles peuvent tous influencer, parfois fortement, le niveau de risque réel ou perçu. Même un virage apparemment simple peut devenir délicat si ces facteurs sont défavorables.
Le système que j’ai conçu s’appuie sur un algorithme volontairement simplifié, il ne prend pas en compte ces variables. Cette approche a l’avantage d’être robuste, rapide et stable dans des contextes variés mais elle présente aussi ses limites : un manque de précision peut apparaître lorsque les conditions réelles diffèrent des hypothèses de calcul. La fiabilité dépend en grande partie de deux paramètres extérieurs : la fréquence de mise à jour de la cartographie et la précision du positionnement GPS. Une carte obsolète ou un signal perturbé peut altérer la qualité des recommandations.
En situation réelle, le comportement du conducteur reste un facteur clé. Certains motards pourraient ignorer l’alerte, par choix, par inattention ou simplement par manque de confiance. À l’inverse, un excès de confiance dans le système peut mener à une dépendance excessive, ce qui réduit la vigilance et les capacités d’analyse face à l’imprévu.
Enfin, un dysfonctionnement ou un retard dans les calculs pourrait fournir une information inadaptée au contexte, ce qui représente un risque, surtout si l’utilisateur se repose entièrement sur l’assistance pour prendre ses décisions.
\vspace{0.5cm}

\commentaire{iot et réprise yann lecun}
Bien que des avancées significatives aient été réalisées depuis plusieurs années grâce aux travaux pionniers de Yann LeCun, figure majeure du deep learning, l’adaptation de ces technologies au domaine de la moto demeure un défi particulièrement complexe. En effet, automatiser la prise de décision en temps réel, déterminer si une trajectoire est sécurisante ou non implique une précision extrême et une capacité d’analyse rapide dans un environnement hautement variable.
Prévoir une trajectoire « sécurisante » ne se limite pas à calculer un simple tracé : cela nécessite de prendre en compte une multitude de paramètres.
Aujourd’hui, de nombreux chercheurs soulignent que la complexité et la variabilité de ces conditions rendent cette anticipation quasiment impossible avec les technologies actuelles. Le moindre retard dans le traitement ou la moindre approximation dans la modélisation pourrait entraîner une recommandation inadaptée, voire dangereuse.
\vspace{0.5cm}

\commentaire{Conclusion}
Le développement d’un système d’assistance à la conduite basé sur la courbure routière et la géolocalisation soulève à la fois des enjeux techniques, légaux et éthiques. Pour être viable, un tel dispositif doit impérativement respecter la réglementation en vigueur, protéger la vie privée des utilisateurs et offrir une robustesse suffisante pour fonctionner en conditions réelles.
Le programme développé fonctionne correctement dans la majorité des situations. Toutefois, certaines limites persistent. Par exemple, la récupération des données relatives aux limitations de vitesse n’est pas toujours fiable. En conséquence, une valeur par défaut de 80 km/h est appliquée. Sur autoroute, où la limitation est de 130 km/h, la courbure calculée est généralement très faible, ce qui ne déclenche aucune recommandation particulière. Cependant, sur des voies rapides limitées à 110 km/h, une courbure suffisamment prononcée peut entraîner l’émission d’une alerte à 80 km/h. Ce choix résulte d’un compromis assumé, visant à privilégier la prudence et à garantir la sécurité du conducteur.
Du point de vue des utilisateurs, certains motards expérimentés peuvent percevoir cette assistance comme trop « prévoyante ». En effet, plus un motard accumule d’expérience et diversifie ses trajets, plus il développe une capacité d’anticipation naturelle face aux dangers. Pour ces profils aguerris, l’alerte peut donc sembler déclenchée trop tôt.
Pour répondre à cette critique, il serait pertinent d’intégrer une variable personnalisable permettant de définir différents niveaux d’alerte, ajustés selon l’expérience ou les préférences du motard. L’accélération pourrait également constituer un paramètre déclencheur : une variation brutale avant un virage pourrait par exemple générer une alerte anticipée, renforçant ainsi la prévention des risques.
Enfin, il convient de souligner que le programme actuel \emph{(Figure~\ref{fig:cartepoints})} s’appuie uniquement sur les données disponibles à l’instant présent. Il ne prend pas encore en compte la trajectoire prévisionnelle ni la direction future du véhicule. L’intégration d’une dimension prédictive dans une version ultérieure permettrait d’améliorer considérablement la pertinence des recommandations et d’anticiper avec plus de précision les situations à risque.