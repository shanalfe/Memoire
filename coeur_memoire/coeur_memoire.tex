\section{Mémoire}

\subsection{Pratique de la route}
Ayant également la casquette d'un usager de la route, j'ai pu expérimenter, pratiquer le deux roues sous plusieurs aspects : \
\begin{itemize}
    \item Les trajets quotidiens,
    \item Les balades entre ami(e)s,
    \item Les road trip de plus d'une dizaine de jours (environ 200 à 300 kms par jour), en montagne.
\end{itemize}

Après plus de 35 000 kms en deux ans, voici les quelques points que j'ai pu observer dans mon entourage.
Le danger des trajets quotidiens, c'est en effet la routine. Les usagers connaissent par coeur la route, mais les dangers sont toujours présents (conditions de route, autres usagers...).
Concernant les balades entre amis, l'effet de groupe à tendance à surestimer ses capacités. Des vitesses excessives peuvent être atteintes, des mauvaises prises de décisions à la suite d'action de certaines personnes peuvent être prise... L'analyse personnelle du motard peut être faussée.
Enfin les road trip, c'est beaucoup de fatigue. La fatigue augmente le temps de réaction. Il faut absolument avoir toutes ses capacités pour pouvoir réagir à toutes situations.
La plupart des usagers lorsque la route est mouillé, la prudence est au maximal (vitesse réduite, peu d'angle, anticipation).

\
Adapter l'IOT représente un défis car