\section{Mémoire}
\commentaire{Idées en cours ainsi que re formulation}


Plusieurs IOT existent sur les voitures et fonctionnent parfaitement sur les motos comme l'angle mort, l'ABS, l'anti-patinage, le contrôle de traction, etc. Cependant, il existe des différences notables entre les deux types de véhicules. Dans cette partie, je vais me consacrer sur l'analyse de la route et l'optimisation des trajectoires.

% faire des schémas avec les trajectoires pour souvelver les problèmes

\subsection{Pratique de la route - Analyse comparative des besoins de sécurité entre voitures et motos}

\commentaire{\\
Analyse des différences fondamentales : protection physique, stabilité, visibilité, comportements routiers.\\
	•	Quelles fonctionnalités IoT des voitures sont difficilement transférables ?\\
•	Quelles fonctionnalités sont transférables mais nécessitent adaptation ?\\
•	Mise en évidence des lacunes spécifiques aux motos.
}


\commentaire{reformulation à faire}
Ayant également la casquette d'un usager de la route, j'ai pu expérimenter, pratiquer le deux roues sous plusieurs aspects : \
\begin{itemize}
    \item Les trajets quotidiens,
    \item Les balades entre ami(e)s,
    \item Les road trip de plus d'une dizaine de jours (environ 200 à 300 kms par jour), en montagne.
\end{itemize}

Après plus de 35 000 kms en deux ans, voici les quelques points que j'ai pu observer dans mon entourage.
Le danger des trajets quotidiens, c'est en effet la routine. Les usagers connaissent par coeur la route, mais les dangers sont toujours présents (conditions de route, autres usagers...).
Concernant les balades entre amis, l'effet de groupe à tendance à surestimer ses capacités. Des vitesses excessives peuvent être atteintes, des mauvaises prises de décisions à la suite d'action de certaines personnes peuvent être prise... L'analyse personnelle du motard peut être faussée.
Enfin les road trip, c'est beaucoup de fatigue. La fatigue augmente le temps de réaction. Il faut absolument avoir toutes ses capacités pour pouvoir réagir à toutes situations.
La plupart des usagers lorsque la route est mouillé, la prudence est au maximal (vitesse réduite, peu d'angle, anticipation).

\vspace{0.5cm}
%TRAJECTOIRES SECURITÉ
\commentaire{idée : montrer que c'est complexe}
Voici des illustrations des trajectoires de sécurité. C'est un thème qui est très abordé par la sécurité routière, les gendarmes ainsi que les usagers des deux-roues. 
\begin{figure}[H]
    \centering
    \includegraphics[width=0.4\textwidth]{coeur_memoire/schéma/trajectoire_sécurité_1.png} 
    \caption{Trajectoire de sécurité utilisée sans obstacle}
\end{figure}
L'objectif ici est de s'ouvrir la visibilité

Voici maintenant la trajectoire idéale dès qu'il y a un autre usager sur la route.
\begin{figure}[H]
    \centering
    \includegraphics[width=0.4\textwidth]{coeur_memoire/schéma/trajectoire_sécurité_4.png} 
    \caption{Trajectoire de sécurité utilisée avec un autre usager}
\end{figure}
Instinctivement, le motard va se rapprocher à l'extérieur du virage pour s'éloigner du danger.
Que se passe-t-il si la route ne le permet pas. Ci-dessous le même virage avec des obstacles, des dégradations de la route qui rend l'impossibilité la trajectoire parfaite ou ne permet pas de continuer la trajectoire en toute sécurité représenté en bleu. 
\begin{figure}[H]
    \centering
    \includegraphics[width=0.4\textwidth]{coeur_memoire/schéma/trajectoire_sécurité_2.png} 
    \caption{Autre configuration de la route avec plusieurs autres dangers}
\end{figure}
Ajoutons maintenant la trajectoire idéale:
\begin{figure}[H]
    \centering
    \includegraphics[width=0.4\textwidth]{coeur_memoire/schéma/trajectoire_sécurité_3.png} 
    \caption{Trajectoire de sécurité utilisée avec des dangers sur la route}
\end{figure}
Cette trajectoire permet d'améliorer l'adhérence des pneus, cependant, elle expose fortement au danger pouvant arriver en face. En effet, si un véhicule arrive en face, le motard n'a pas le temps de réagir et de se décaler. Il est donc important de prendre en compte les conditions de la route, la visibilité et les autres usagers pour adapter sa trajectoire. Pour pouvoir ajuster la trajectoire, la vitesse de déplacement est à prendre en compte. En effet, plus la moto va vite et plus la trajectoire sera difficile à ajuster.

Voici d'autres exemples avec plusieurs solutions, avec chacune leur avantages et inconvénients.

\begin{figure}[H]
  \centering
  
  \begin{subcaptionbox}{Trajectoire possible 1\label{fig:img1}}[0.4\linewidth]
    {\includegraphics[width=\linewidth]{coeur_memoire/schéma/trajectoire_sécurité_7.png}}
  \end{subcaptionbox}
  \hfill
  \begin{subcaptionbox}{Trajectoire possible 2\label{fig:img2}}[0.4\linewidth]
    {\includegraphics[width=\linewidth]{coeur_memoire/schéma/trajectoire_sécurité_6.png}}
  \end{subcaptionbox}
  
  \vspace{0.5cm}
  
  \begin{subcaptionbox}{Trajectoire possible 3\label{fig:img3}}[0.4\linewidth]
    {\includegraphics[width=\linewidth]{coeur_memoire/schéma/trajectoire_sécurité_5.png}}
  \end{subcaptionbox}
  
  \caption{Autres exemples de trajectoires de sécurité}
  \label{fig:mosaique}
\end{figure}

Commentons :\
\begin{itemize}
    \item La trajectoire 1 est la plus sécurisée. Elle permet de s'éloigner de la voiture arrivant en face. Cependant, une vitesse trop élevée ne permettra de se rabattre à l'intérieur du virage comme nous l'indique la trajectoire verte.
    \item La trajectoire 2 est plus risquée, car elle rapproche le motard du danger. Ce virage peut être passé plus vite car on observe une belle courbe.
    \item La trajectoire 3 est relativement correcte. Elle éloigne du danger arrivant en face, cependant, rester trop vers le bas côté peut être dangereux. En effet, si un obstacle (mauvais état du bas côté, animal...) se trouve sur le bas côté, le motard n'a pas le temps de réagir et de se décaler. De plus, la visibilité est très réduite.
\end{itemize}
Pour conclure sur ces schémas, il existe beaucoup de solutions mais peu adaptée spécifiquement au motard. Certains sont plus "casse-coup" que d'autres, l'expérience... C'est un vecteur à prendre en compte.

\todo{ajout des réponses d'une enquête partagé sur les réseaux, expériences etc...}

\todo{Ouverture sur la complexité + énonciation des différents vecteurs}

Le défi ici c'est de pouvoir adapter la trajectoire en temps réel, en fonction de la vitesse, de l'angle d'inclinaison, des conditions de la route et des autres usagers. Les technologies IoT peuvent jouer un rôle clé dans cette adaptation en fournissant des données en temps réel sur l'environnement et en permettant une communication entre les différents usagers de la route. La meilleure trajectoire sera celle où le motard se sentira le plus  en sécurité. Cela dépend de son expérience, de ses capacités cependant il y a beaucoup de possibilités.


%PLAISIR
\vspace{0.5cm}
Le plaisir de conduire une moto réside dans la sensation de liberté qu'elle procure. Cependant, cette liberté s'accompagne de responsabilités, notamment en matière de sécurité. Les technologies IoT peuvent contribuer à améliorer cette sécurité tout en préservant le plaisir de conduite.

\vspace{0.5cm}
Adapter l'IOT représente un défis car il faut prendre en compte de nombreux paramètres (environnement, comportement des usagers, etc.) et s'assurer que les solutions proposées sont adaptées à chaque situation.

\newpage
\subsection{ Étude critique des technologies IoT existantes (voitures vs motos)}
\commentaire{\\
avec un regard critique et des cas concrets.\\
Analyse de cas réels où les technologies IoT ont été adaptées à la moto (Airbags connectés, gilets intelligents, casques IoT, etc.).\\
	Freins techniques : taille réduite, alimentation électrique, exposition météo, vibrations, etc.
}

\subsection{ Propositions d’adaptations technologiques}
\commentaire{	\\
•	Systèmes de communication V2X (Vehicle-to-Everything) pour motos : miniaturisation, portabilité, est ce que cela est possible ? les conditions.\\
	•	Capteurs embarqués sur la moto et sur le pilote (gilet, casque, smartphone).\\
	•	Intégration d’IA pour la détection du risque en temps réel : freinage d’urgence, angle d’inclinaison, anticipation de collision. => a voir\\
	•	Systèmes d’alerte connectés avec autres usagers (voitures, infrastructures).\\
	•	Possibilité d’un écosystème IoT dédié aux motos, interopérable avec celui des voitures. => à voir si ca existe déja
}

\subsection{ Étude de faisabilité et limites}
\commentaire{\\
    •	Quels obstacles (coût, poids, énergie, connectivité, acceptabilité des motards) à l’implémentation ?\\
	•	Quelles pistes pour la recherche ou le développement industriel ?\\
	•	(une maquette fonctionnelle, un prototype conceptuel, ou même une étude de cas simulée)\\
    •	ouverture défi environnemental}

\subsection{Apport personnel et positionnement}
\commentaire{\\
	•	proposer qqch de nouveau ou différemment par rapport à l’existant.
•	Une vision d’ensemble : comment la réflexion contribue à combler le fossé technologique entre sécurité voiture vs moto.
}
