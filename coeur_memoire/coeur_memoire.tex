\section{Mémoire}
\commentaire{Idées en cours ainsi que re formulation}
De nombreux systèmes IoT déjà présents sur les voitures trouvent également leur place sur les motos, tels que l’alerte d’angle mort, l’ABS, l’anti-patinage ou encore le contrôle de traction. Toutefois, malgré ces points communs, des différences significatives subsistent entre ces deux types de véhicules, notamment dans leur comportement et leurs contraintes spécifiques. Dans cette partie, je vais concentrer mon analyse sur l’évaluation de la route et sur les moyens d’optimiser la prise de trajectoire.

\subsection{Pratique de la route - Analyse comparative des besoins de sécurité entre voitures et motos}

\commentaire{\\
Analyse des différences fondamentales : protection physique, stabilité, visibilité, comportements routiers.\\
	•	Quelles fonctionnalités IoT des voitures sont difficilement transférables ?\\
•	Quelles fonctionnalités sont transférables mais nécessitent adaptation ?\\
•	Mise en évidence des lacunes spécifiques aux motos.
}

J’ai eu l’occasion d’expérimenter la pratique du deux-roues sous différents angles :\
\begin{itemize}
    \item Les trajets du quotidien,
    \item Les balades entre ami(e)s,
    \item Les road trip de plus d'une dizaine de jours (environ 200 à 300 kms par jour), souvent en milieu montagneux.
\end{itemize}
\vspace{0.5cm}

Avec plus de 35 000 km parcourus en deux ans, voici les constats que j’ai pu faire autour de moi :
\begin{itemize}
  \item Les trajets du quotidien peuvent paraître anodins mais c’est justement la routine qui les rend dangereux. En connaissant la route par cœur, on a tendance à relâcher sa vigilance, alors que les risques restent bien réels (état de la chaussée, comportement imprévisible des autres usagers, etc.).
  \item Les balades entre ami(e)s apportent un vrai plaisir de conduite mais l’effet de groupe peut parfois inciter à dépasser ses limites. On peut se retrouver à rouler à des vitesses inadaptées ou à prendre de mauvaises décisions sous l’influence de comportements plus audacieux. Le jugement individuel peut alors être altéré.
  \item Les road trips, quant à eux, demandent une grande endurance. La fatigue s’accumule rapidement, et avec elle, le temps de réaction s’allonge. Il est essentiel d’être pleinement en possession de ses capacités pour pouvoir réagir correctement en cas de situation imprévue.
\end{itemize}
\vspace{0.5cm}

Enfin, un point important, lorsque la chaussée est mouillée, la majorité des motards adoptent naturellement une conduite plus prudente, une vitesse réduite, une prise d’angle limitée et meilleure anticipation. Cela montre que la perception du risque influence fortement le comportement.
Abordons maintenant une difficulté que beaucoup de deux-roues rencontrent : Les virages.

\subsubsection{Étude des virages}
\commentaire{TRAJECTOIRES SECURITÉidée : montrer que c'est complexe}
Les illustrations ci-dessous représentent différentes trajectoires dites de sécurité. Ce sujet occupe une place centrale dans les campagnes de prévention menées par les acteurs de la sécurité routière \emph{(voir Figure~\ref{fig:trajectoire_securite})}. Il est régulièrement abordé par les forces de l’ordre, notamment les gendarmes spécialisés dans la sécurité des deux-roues motorisés mais également par les formateurs, les auto-écoles et les usagers eux-mêmes.\\
En effet, la trajectoire de sécurité constitue une technique de conduite essentielle pour limiter les risques en virage, optimiser la visibilité et mieux anticiper les éventuels dangers. Elle permet au motard d’adopter une position plus stratégique sur la chaussée, en tenant compte à la fois du tracé de la route, de l’environnement et de la circulation en sens inverse.\\
La sensibilisation à cette pratique est donc fortement encouragée, que ce soit lors des formations initiales, des stages post-permis ou à travers les communications des institutions publiques. Les illustrations présentées ici ont pour but de mieux comprendre ces trajectoires et de visualiser les choix possibles selon différents contextes routiers.
\begin{figure}[H]
    \centering
    \includegraphics[width=0.4\textwidth]{coeur_memoire/schéma/trajectoire_sécurité_1.png} 
    \caption{Trajectoire de sécurité utilisée sans obstacle.}
\end{figure}
Adopter une trajectoire comme celle présentée ci-dessus permet avant tout d’améliorer la visibilité à l’entrée et au cœur du virage. En s’écartant légèrement de l’intérieur de la courbe, le motard élargit son champ de vision, ce qui facilite l’anticipation d’éventuels obstacles, de zones à faible adhérence ou encore de la présence d’autres usagers. Cependant, cette trajectoire doit être ajustée lorsqu’un véhicule arrive en sens inverse. Dans ce cas, la priorité n’est plus seulement la visibilité, mais aussi la sécurité du croisement. Il devient alors nécessaire d’adopter une trajectoire dite “de compromis” ou “de sécurité”, qui conserve une marge de manœuvre tout en maintenant une distance suffisante avec l’usager opposé.\\
La figure suivante illustre cette trajectoire adaptée, considérée comme idéale lorsqu’un autre véhicule est présent sur la route. Elle permet un passage fluide, sans chevauchement de la voie adverse, tout en garantissant la stabilité du deux-roues dans la courbe. Ce type d’ajustement reste essentiel pour limiter les risques de collision frontale, en particulier dans les virages où la visibilité est réduite.

\begin{figure}[H]
    \centering
    \includegraphics[width=0.4\textwidth]{coeur_memoire/schéma/trajectoire_sécurité_4.png} 
    \caption{Trajectoire de sécurité utilisée avec un autre usager.}
\end{figure}
Instinctivement, le motard va se rapprocher de l'intérieur du virage pour s'éloigner du danger représenté en bleu par la voiture.
Pour poursuivre cette démonstration, nous allons y ajouter d'autres dangers sur la route représentés par des objets en bleu rendant l'impossibilité de prendre une trajectoire "parfaite". Dans la vie courante, cela peut représenter des gravillons, un animal mort sur la route, des plaques d'égout, des nids de poule, des bandes d'étanchéité (mastics), etc. Ces facteurs compromettent instantanément l’adhérence des pneumatiques, ce qui peut entraîner une perte de contrôle et, potentiellement, une chute.
\begin{figure}[H]
    \centering
    \includegraphics[width=0.4\textwidth]{coeur_memoire/schéma/trajectoire_sécurité_2.png} 
    \caption{Autre configuration de la route avec plusieurs autres dangers.}
\end{figure}
Ajoutons maintenant la trajectoire idéale à la situation permettant de garder l'adhérence des pneus:
\begin{figure}[H]
    \centering
    \includegraphics[width=0.4\textwidth]{coeur_memoire/schéma/trajectoire_sécurité_3.png} 
    \caption{Trajectoire de sécurité utilisée avec des dangers sur la route.}
    \label{fig:trajectoire_securite_difficulte}
\end{figure}
Cette trajectoire améliore l’adhérence des pneus mais elle présente un risque important en cas de danger venant en sens inverse. Si un véhicule surgit en face, le motard dispose de très peu de temps pour réagir ou se décaler, ce qui peut entraîner un accident. Il est donc essentiel d’adapter sa trajectoire en fonction de plusieurs éléments : l’état de la route, la visibilité et la présence d’autres usagers. La vitesse joue également un rôle déterminant : plus la moto roule vite, plus il devient difficile de corriger la trajectoire à temps. Voici ci-dessous plusieurs exemples de trajectoires, chacune présentant ses avantages et ses limites.
\begin{figure}[H]
  \centering
  \begin{subcaptionbox}{Trajectoire possible 1.}[0.4\linewidth]
    {\includegraphics[width=\linewidth]{coeur_memoire/schéma/trajectoire_sécurité_7.png}}
  \end{subcaptionbox}
  \hfill
  \begin{subcaptionbox}{Trajectoire possible 2.}[0.4\linewidth]
    {\includegraphics[width=\linewidth]{coeur_memoire/schéma/trajectoire_sécurité_6.png}}
  \end{subcaptionbox}
  \vspace{0.5cm}
  
  \begin{subcaptionbox}{Trajectoire possible 3.}[0.4\linewidth]
    {\includegraphics[width=\linewidth]{coeur_memoire/schéma/trajectoire_sécurité_5.png}}
  \end{subcaptionbox}
  \caption{Autres exemples de trajectoires de sécurité.}
\end{figure}

Analysons et commentons ces trajectoires :
\begin{itemize}
    \item Trajectoire 1 : C’est la plus sécurisante. Elle permet de s’éloigner efficacement du véhicule venant en sens inverse. Toutefois, si la vitesse est trop élevée, il sera difficile de revenir à l’intérieur du virage, comme le montre la trajectoire en vert.
    \item Trajectoire 2 : Elle est plus risquée, car elle place le motard plus près du danger potentiel. Même si la courbe semble fluide et permet une prise de virage à vitesse plus élevée, elle réduit la marge de manœuvre en cas d’imprévu.
    \item Trajectoire 3 : Elle représente un bon compromis, car elle maintient une certaine distance avec les véhicules en face. Cependant, rester trop proche du bas-côté peut s’avérer dangereux, notamment en cas d’obstacle imprévu (dégradation de la chaussée, présence d’un animal, etc.). La visibilité y est également plus restreinte, ce qui peut compromettre l’anticipation, l'analyse du virage.
\end{itemize}
Pour conclure sur ces schémas, il existe de nombreuses trajectoires possibles mais aucune n’est parfaite. Certaines sont plus risquées que d’autres, et l’expérience du pilote joue un rôle clé dans le choix et la gestion de la trajectoire. Cela met en évidence à quel point cette phase de conduite est exigeante et complexe et combien de facteurs doivent être pris en compte pour envisager une assistance efficace.


%ETUDE SONDAGE
\subsubsection{Enquête auprès de motards - Recueil de ressenti}
Afin de mieux répondre la problématique, j’ai réalisé une enquête auprès de plusieurs groupes de motards sur les réseaux sociaux, dans le but de recueillir leur ressenti. Comprendre le point de vue des principaux concernés est en effet primordial pour une analyse pertinente. Cela m'a permis de mieux comprendre leurs besoins, leurs attendus afin d'avoir un œil beaucoup plus objectif de ce que moi je peux vivre, ressentir.. L’échantillon retenu comprend une vingtaine de participants, avec une répartition de 58,8 \% d'hommes contre 41,2 \% de femmes.
Voici les différents profils des participants :
\begin{figure}[H]
  \centering
  \begin{subcaptionbox}{Age des participants}[0.5\linewidth]
    {\includegraphics[width=\linewidth]{coeur_memoire/graphique/age.png}}
  \end{subcaptionbox}
  \hfill
  \begin{subcaptionbox}{Nombre d'années de permis}[0.5\linewidth]
    {\includegraphics[width=\linewidth]{coeur_memoire/graphique/nb_annees_permis.png}}
  \end{subcaptionbox}

  \begin{subcaptionbox}{Nombre de kms à l'année}[0.5\linewidth]
    {\includegraphics[width=\linewidth]{coeur_memoire/graphique/nb_km_an.png}}
  \end{subcaptionbox}
  \hfill
  \begin{subcaptionbox}{Utilisation de la moto}[0.5\linewidth]
    {\includegraphics[width=\linewidth]{coeur_memoire/graphique/utilisation_moto.png}}
  \end{subcaptionbox}
  \caption{Profils des participants.}
\end{figure}
Pour conclure, l’enquête a permis de recueillir les témoignages de motards aux profils variés. Une majorité d’entre eux parcourt moins de 3 000 km par an, ce qui représente un faible kilométrage et donc une expérience relativement limitée sur la route. Par ailleurs, la plupart des répondants possèdent des deux-roues de type roadster, principalement dans des cylindrées comprises entre 400 cc et 600 cc. On retrouve des Honda, Kawasaki, Aprillia, KTM et Yamaha. Ce sont des motos qui sont inférieures à 10 000 euros, neuves chez le concessionnaire.
Au cours de mes derniers voyages dans les Alpes (Suisse, Italie et France), et en Autriche, j’ai observé une forte présence de motos BMW GS\footnote{Gelände/Straße : tout-terrain/route. C'est une moto conçue pour être à la fois confortable sur route et capable de rouler sur des chemins non goudronnés. Premiers prix : 12 000 euros, haut de gamme : environ 30 000 euros.}. J’estime que ces modèles représentaient entre 50\% et 70\% des deux-roues rencontrés sur les routes de montagne. Ce sont des motos robustes et confortables pour les voyages pouvant embarquer beaucoup de technologies. \\
Voici quelques retours d'expérience des motards en situation d'urgence selon l'enquête :\\
\begin{itemize}
  \item "Proche d'un rond-point, j'ai mal évalué la distance. Donc je suis arrivée trop vite, j'étais déjà dans l'insertion sur le rond-point et j'ai failli terminer dans un véhicule. Contrainte d'effectuer un freinage d'urgence et de me mettre le plus à l'extérieur pour éviter la catastrophe."
  \item "La voiture de La Poste stationnée en plein virage en sens inverse, empiétant sur ma voie. Sur une route à 50 km/h en agglomération. J'ai dû pratiquer l'évitement pour ne pas me percuter le véhicule."
  \item "Ma roue s’est coincée dans un rail de tram .. "
  \item "J'ai pris un trou (pas visible avant d'être dedans) sur une nationale, ce qui m'a fait guidonner, impossible à rattraper donc la moto a fini par se coucher et une belle glissade pour finir."
  \item "Évitement de personne qui tourne sans avoir mis de clignotant. Classique."
\end{itemize}
\vspace{0.5cm}

Pour la fin de l'étude, j'ai demandé aux participants quelles étaient leurs attentes concernant les technologies IoT sur les motos. Certains ne souhaitent pas de technologie IoT, car ils estiment que cela peut nuire à la conduite et à la sensation de liberté. D'autres sont favorables à l'intégration de technologies IoT. Par exemple, un assistant virtuel qui indiquerait au pilote les dangers potentiels sur la route (chaussée dégradée, virage dangereux, gravillons...). Les autres retours sont des technologies qui sont déjà présentes sur les motos du marché comme l'ABS, le contrôle de traction, l'anti-patinage, etc. Cependant, d'après certains motards, il faut mettre ces éléments de série et non en option comme l'appel d'urgence.
\vspace{0.5cm}

Les réseaux sociaux témoignent régulièrement d’accidents impliquant des motards, rappelant à quel point ils restent des usagers particulièrement exposés. Par exemple, un article \emph{(Voir Article~\ref{articlemoto})} récent publié par Ouest-France rapporte qu'une jeune femme de 26 ans a perdu tragiquement la vie lors d'une leçon de moto-école dans l’Ain. L'accident s'est produit le samedi 2 août 2025 près de Châtillon-la-Palud. Alors qu'elle circulait sur une portion de route comportant plusieurs virages, elle a perdu le contrôle de son véhicule et percuté violemment une glissière de sécurité. La jeune femme n’a pas survécu à ses blessures. Aucun autre véhicule ne serait impliqué. Ce tragique événement rappelle une fois de plus l’importance d'améliorer constamment la sécurité des motards et l'importance de savoir bien aborder un virage.
\vspace{0.5cm}

Le véritable défi réside dans la capacité à adapter la trajectoire en temps réel, en fonction d’un ensemble de paramètres dynamiques et interdépendants. Cette adaptation ne dépend pas uniquement de la courbure de la route mais également de nombreux vecteurs qui rendent la situation complexe à modéliser et à anticiper. Parmi ces vecteurs, on peut citer :\\
\begin{itemize}
\item La vitesse instantanée du véhicule, qui influence directement la trajectoire possible et le temps de réaction du conducteur ;
\item L’angle d’inclinaison de la moto, déterminant l’équilibre et l’adhérence lors de la prise de virage ;
\item L’état de la chaussée, incluant l’adhérence, les irrégularités, les débris ou l’humidité ;
\item Les conditions météorologiques, telles que la pluie, le vent, la chaleur ou le brouillard, qui impactent à la fois la visibilité et la tenue de route ;
\item La présence et le comportement des autres usagers, qu’il s’agisse de véhicules motorisés, de cyclistes ou de piétons ;
\item La géométrie de la route, incluant la largeur, le dénivelé, les virages successifs ou les intersections.
\end{itemize}
Dans ce contexte complexe, les technologies IoT (Internet of Things) peuvent jouer un rôle central. En fournissant des données en temps réel sur l’environnement routier, elles permettent de mieux analyser la situation et de faciliter la prise de décision. Par exemple, une communication entre véhicules pourrait alerter un motard d’un freinage brusque à venir, ou un capteur pourrait ajuster une recommandation de trajectoire en fonction de l’adhérence mesurée sur la route.\\
Il est important de rappeler que, malgré les apports technologiques, la trajectoire la plus appropriée reste celle dans laquelle le motard se sent en sécurité. Elle dépend fortement de son expérience, de ses réflexes, de sa confiance en lui et de sa connaissance de la route. Il n’existe donc pas une seule "bonne" trajectoire mais une pluralité d’options, à ajuster en fonction du contexte, des données disponibles et des capacités du pilote.

\subsubsection{Défi passion}
\vspace{0.5cm}
Le plaisir de conduire une moto réside dans la sensation de liberté qu'elle procure. Cependant, cette liberté s'accompagne de responsabilités, notamment en matière de sécurité. Les technologies IoT peuvent contribuer à améliorer cette sécurité tout en préservant le plaisir de conduite. "Passionné auto-moto depuis mon plus jeune âge, j'aime rouler souvent seul mais j'aime me sentir libre et pouvoir aller où je veux et quand je veux, la moto, c'est indescriptible et c'est comme une drogue mais c'est la passion." selon un retour de l'enquête.

\vspace{0.5cm}
L’intégration de l’IoT dans l’univers de la moto représente un défi majeur car elle doit tenir compte de nombreux paramètres : les conditions de l’environnement, les habitudes de conduite et les attentes spécifiques des motards. Les solutions développées doivent être suffisamment flexibles pour s’adapter à des situations variées, tout en préservant ce qui fait l’essence de la moto : le plaisir, la liberté et la passion de la conduite.

\newpage
\subsection{ Étude critique des technologies IoT existantes (voitures vs motos)}
\commentaire{\\
avec un regard critique et des cas concrets.\\
Analyse de cas réels où les technologies IoT ont été adaptées à la moto (Airbags connectés, gilets intelligents, casques IoT, etc.).\\
  Freins techniques : taille réduite, alimentation électrique, exposition météo, vibrations, etc.\\
  lidar et radar ?\\
}

\subsubsection{Exemples d’IoT adaptés et développés pour la moto}
Plusieurs dispositifs connectés ont été développés spécifiquement pour les motards ces dernières années. Certaines technologies se rapprochent de celles utilisées dans l'automobile comme les airbags, la navigation... Voici quelques exemples :
\begin{itemize}
  \item Airbags connectés (par exemple : Dainese Smart Jacket, In\&motion avec Ixon): Ces gilets intelligents détectent les mouvements anormaux (chute, décélération brutale) via des accéléromètres et gyroscopes, et déclenchent un coussin gonflable. L’algorithme est souvent connecté à une plateforme cloud qui analyse les données de milliers d’utilisateurs pour améliorer la détection. Installer directement un airbag sur une moto n’est pas envisageable. Contrairement à une voiture, la moto ne dispose pas d’une structure fermée ni de points d’ancrage solides pour accueillir un dispositif de ce type. En cas de chute, le motard est éjecté de son véhicule, ce qui rend l’efficacité d’un airbag fixe quasiment nulle. C’est pour cette raison que les airbags destinés aux motards ont été pensés sous forme de gilets ou de vestes connectées, capables de se déployer autour du corps du conducteur, quelle que soit la situation,
  \item Casques connectés (par exemple : CrossHelmet, Shoei IT-HT, Livall): Ils embarquent caméras arrière, GPS, commandes vocales, HUD\footnote{affichage tête haute} et parfois même des alertes de proximité de véhicules. Certaines marques permettent la communication entre motards via Bluetooth ou 4G,
  \item Systèmes d’aide à la navigation : Solutions telles que Calimoto, TomTom Ride... offrent une navigation simplifiée, pensée pour les deux-roues, avec une interface minimaliste et résistante aux intempéries.
  \item Les feux additionnels, parfois connectés : Ils constituent une aide précieuse pour renforcer la visibilité, aussi bien pour le motard que pour les autres usagers de la route, en particulier dans des conditions de faible luminosité. Certains modèles intègrent des capteurs capables d’ajuster automatiquement l’intensité lumineuse en fonction de la vitesse de la moto, de son angle d’inclinaison, offrant ainsi un éclairage optimisé et adapté à la situation. Certains modèles peuvent être personnalisés pour répondre à des actions comme des clignotants, des feux de détresse... Toutefois, cette technologie a un coût : les prix varient allant d’environ 50 euros pour les modèles les plus simples à près de 1 000 euros pour les dispositifs les plus avancés haut de gamme.

\end{itemize}
\vspace{0.5cm}
La technologie peut jouer un rôle essentiel pour la gestion des virages nocturnes. Les phares adaptatifs constituent déjà une première solution pour pallier ce problème. Les phares adaptatifs modifient automatiquement l'angle et la puissance du faisceau lumineux en fonction de la vitesse, de l'inclinaison ou du parcours, améliorant ainsi la clarté lors des courbes. Des dispositifs IoT embarqués pourraient aller plus loin en intégrant des capteurs de luminosité, des caméras thermiques ou infrarouges, permettant de détecter les piétons, animaux ou obstacles au-delà du champ des phares classiques. De tels dispositifs offriraient au motard une meilleure capacité à anticiper, diminuant ainsi les dangers associés à la conduite de nuit.
\vspace{0.5cm}

Ces exemples illustrent que l’écosystème IoT appliqué à la moto progresse mais demeure nettement moins mature et avancé que celui dédié à l’automobile. Il serait donc pertinent de mesurer concrètement l’impact de ces technologies sur la sécurité, en s’appuyant sur des données chiffrées. Par exemple, quantifier la baisse du nombre de blessures graves grâce à l’utilisation d’airbags intelligents permettrait de dépasser le simple discours technologique pour appuyer les bénéfices réels.
La question de l’accessibilité reste également centrale. Le coût élevé de certains équipements comme un casque connecté pouvant varier entre 500 et 1500 euros peut représenter un frein pour de nombreux motards. De plus, la complexité liée à l’installation, à la configuration ou à la mise à jour de ces dispositifs peut décourager leur adoption à grande échelle.

\subsubsection{Les radars}
Aujourd'hui, de nombreuses motos sont équipées de technologies directement issues de l’industrie automobile, telles que l’ABS (système antiblocage des roues), la détection d’angle mort ou encore le régulateur de vitesse adaptatif. Ces dispositifs ont été progressivement adaptés pour répondre aux contraintes spécifiques des deux-roues, en tenant compte des enjeux liés à l'équilibre, à la maniabilité et à la sécurité du motard.
Après avoir expérimenté ces technologies, les retours des utilisateurs sont globalement très positifs. Ils mettent notamment en avant le gain en confort, particulièrement sur les longs trajets ou dans les conditions difficiles. Le régulateur adaptatif, par exemple, réduit significativement la fatigue en ajustant automatiquement la vitesse en fonction du véhicule qui précède. La détection d’angle mort accroît quant à elle la vigilance du conducteur, permettant une meilleure anticipation des dangers et réduisant ainsi les risques de collision.\\
L'intégration de ces technologies améliore non seulement l’expérience de conduite mais contribue également de manière concrète à la sécurité routière. Moins sollicités sur des tâches répétitives, les motards conservent une meilleure concentration et sont donc moins exposés aux erreurs liées à la fatigue ou à l’inattention. Ces progrès montrent clairement l'intérêt de continuer à développer et à adapter des technologies automobiles avancées aux besoins spécifiques des utilisateurs de deux-roues motorisés.\\
%LIDAR
Les radars sont aujourd’hui intégrés aux modèles les plus récents de motos et remplissent efficacement leur fonction. Comme mentionné précédemment, la détection des éléments présents sur la route joue un rôle essentiel dans la sécurité du motard, tout comme les actions automatisées ou assistées qui en découlent. Ces technologies permettent notamment d’anticiper certains dangers, d’éviter des collisions ou de réguler la vitesse en fonction de l’environnement immédiat.\\
Cependant, les radars présentent certaines limites. Leur capacité à différencier les types d’objets ou à analyser finement l’environnement reste restreinte. C’est dans ce contexte que l’utilisation du LiDAR (Light Detection and Ranging) s’avère particulièrement intéressante. Cette technologie repose sur l’émission de faisceaux laser pour mesurer avec grande précision les distances et modéliser l’environnement en trois dimensions. Elle est très performante pour la cartographie 3D et la classification des objets, ce qui en fait un atout majeur dans les systèmes d’aide à la conduite avancés.\\

Néanmoins, le LiDAR n’est pas sans contraintes. Il reste très sensible aux conditions météorologiques défavorables telles que la pluie, le brouillard ou la neige, qui peuvent altérer la qualité des données recueillies. De plus, son intégration sur une moto pose plusieurs défis pratiques. La taille et le poids des capteurs peuvent affecter la maniabilité, l’esthétique et l’ergonomie de la moto. Sans compter le coût : en 2020, selon PW Consulting\cite{marche_capteur_lidar}, un capteur LiDAR compact se situait entre 3 000 et 10 000 euros, un prix difficilement envisageable pour la majorité des motards.\\

Heureusement, les coûts tendent à diminuer avec l’évolution du marché et la démocratisation de cette technologie. Aujourd’hui, il est possible de se procurer un capteur LiDAR pour un montant compris entre 500 et 2 000 euros, en fonction des gammes et des performances souhaitées. Cela ouvre de nouvelles perspectives pour une intégration progressive sur les deux-roues, à condition que les autres contraintes techniques soient également maîtrisées.\\

\vspace{0.5cm}
Les prototypes :\\
\begin{table}[ht]
\centering
\begin{tabular}{|l|c|}
  \hline
  Éléments & Prix \\
  \hline
  Capteurs LIDAR & 500 à 2 000 euros \\
  Traitement embarqués & 300 à 1 500 euros \\
  Interfaces et boîtier durci & 100 à 400 euros \\
  Intégration et calibration & 200 à 1 000 euros \\
  Total estimation & 1 100 à 5 900 euros \\
  \hline
\end{tabular}
\caption{Prix moyen estimé pour la technologie LIDAR sur moto (2025).}
\end{table}

Prenons les coûts moyens d'un deux roues :\\
\begin{table}[ht]
\centering
\begin{tabular}{|l|c|c|}
\hline
\textbf{Catégorie} & \textbf{Cylindrée} & \textbf{Prix moyen (TTC, €)} \\
\hline
Scooter & 50--125 cm³ & 2\,000 -- 4\,500 euros \\
Moto légère & 125 cm³ & 3\,500 -- 5\,000 euros \\
Moto moyenne cylindrée & 300--650 cm³ & 6\,000 -- 9\,000 euros \\
Moto grosse cylindrée & 700--1 000+ cm³ & 10\,000 -- 18\,000 euros \\
Moto sportive haut de gamme & 1 000+ cm³ & 18\,000 -- 30\,000 euros \\
Moto électrique & équiv. 50--125 cm³ & 4\,000 -- 12\,000 euros \\
\hline
\multicolumn{2}{|c|}{\textbf{Prix moyen toutes catégories}} & \textbf{7\,000 -- 9\,000 euros} \\
\hline
\end{tabular}
\caption{Prix moyen estimé des deux-roues neufs en 2025 (France/Europe) par The Market Reports.}
\end{table}

Pour conclure, le coût de la technologie LIDAR appliquée aux deux-roues représente une part très importante du prix total d’une moto neuve. Selon les estimations, son intégration varie entre 12\% et 84\% du prix moyen d’un deux-roues (7 000 à 9 000 €). Cela signifie que pour certains modèles, le prix du capteur peut représenter presque autant que la moto elle-même. Ce surcoût constitue donc un frein majeur à la généralisation de la technologie, surtout dans un marché où le critère prix reste déterminant pour de nombreux usagers.


\subsubsection{GPS}
Aujourd’hui, les systèmes GPS intégrés aux voitures ont atteint un haut niveau de maturité. Ils offrent une expérience fluide, fiable et parfaitement adaptée aux besoins de la conduite automobile. Les GPS modernes embarquent des composants hautement performants : un récepteur GNSS\footnote{Global Navigation Satellite System : capte les signaux émis par les satellites GPS, Galileo, GLONASS ou BeiDou pour calculer la position.}, des antennes optimisées, ainsi que des processeurs et chipsets de navigation capables d’interpréter rapidement les données satellites et d’exécuter des algorithmes de positionnement précis. L’ensemble est complété par des haut-parleurs intégrés, permettant de recevoir des instructions vocales sans quitter la route des yeux.
La situation est bien différente pour les motos. Ici, aucun habitacle ne protège contre la pluie, la poussière, les vibrations ou les chocs. Un GPS moto doit donc être conçu pour résister à ces contraintes. Cela implique l’utilisation d’un boîtier étanche et renforcé, conforme aux normes IP67/IP68\footnote{Les indices IP67 et IP68, définis par la norme IEC 60529, indiquent le niveau de protection d’un appareil contre la pénétration de poussière et d’eau.}, ainsi qu’un système de fixation robuste pour supporter les conditions de roulage. Les vibrations du moteur et de la route peuvent en effet détériorer des composants qui, dans un environnement automobile stable, ne subiraient aucun dommage.

\subsubsection{Contraintes techniques et comparaison voitures vs motos dans l’adoption des technologies IoT}

L’intégration de solutions IoT diffère fortement entre voitures et motos, en raison de contraintes physiques, énergétiques et environnementales propres à chaque type de véhicule. Le tableau \ref{tab:comparaison_iot_voiture_moto} présente une vue d’ensemble, suivie d’une analyse détaillée des points clés pour les deux-roues.

\begin{table}[H]
\centering
\label{tab:comparaison_iot_voiture_moto}
\renewcommand{\arraystretch}{1.15}
\begin{tabular}{|p{3.2cm}|p{6.5cm}|p{6.5cm}|}
\hline
\textbf{Critère} & \textbf{Voiture} & \textbf{Moto} \\
\hline
\textbf{Espace disponible} &
Suffisant pour intégrer de nombreux composants électroniques (capteurs, unités de calcul, caméras, etc.) &
Très limité, surtout sur les modèles sportifs ou compacts ; intégration plus contraignante \\
\hline
\textbf{Alimentation électrique} &
Batterie de forte capacité alimentant de multiples systèmes &
Batterie réduite, imposant des dispositifs sobres en énergie \\
\hline
\textbf{Protection matérielle} &
Composants protégés par un habitacle fermé &
Composants exposés aux intempéries, à la poussière, aux vibrations et aux variations thermiques \\
\hline
\textbf{Sécurité passive} &
Ceintures, airbags frontaux et latéraux, zones de déformation &
Gilet airbag externe, casque et protections mécaniques personnelles \\
\hline
\textbf{Ergonomie d’affichage} &
Écrans tactiles, HUD, commandes vocales &
Affichage minimaliste via smartphone, casque connecté ou retour haptique/sonore \\
\hline
\textbf{Maturité des technologies IoT} &
Avancée (ADAS, V2X, LIDAR, conduite autonome partielle) &
Émergente (casques connectés, gilets intelligents, radars adaptatifs) \\
\hline
\end{tabular}
\caption{Comparaison des conditions d’intégration des technologies IoT entre voitures et motos.}
\end{table}

Pour les motos, la compacité des composants est un enjeu majeur : chaque capteur, antenne ou processeur doit être discret pour préserver l’esthétique et la maniabilité. La faible capacité électrique impose l’utilisation de modules économes en énergie. Les équipements doivent également résister à un environnement exigeant : exposition directe à la pluie, aux rayons UV, à la poussière et aux projections, avec un indice de protection d’au moins IP67 pour garantir la fiabilité.
Les vibrations et chocs, dus au moteur et à l’état de la route, nécessitent des fixations robustes et des matériaux résistants à l’usure. L’ergonomie est aussi déterminante : l’affichage des données, qu’il soit intégré au tableau de bord ou au casque, doit rester lisible et non distrayant. Ces contraintes cumulées expliquent que, malgré un fort potentiel en matière de sécurité et de confort, certaines innovations IoT restent encore peu démocratisées sur deux-roues.


\subsubsection{Pistes pour intégrer l’IoT dans les deux-roues}
Pour surmonter les limites actuelles, plusieurs pistes peuvent être envisagées : 
\begin{itemize}
  \item Miniaturisation : développer des capteurs IoT plus petits et moins gourmands, spécifiques aux motos,
  \item Énergie autonome : utiliser des solutions rechargeables ou solaires intégrées aux équipements (casque, gilet),
  \item Normes de résistance : généraliser des composants certifiés IP67 ou supérieurs,
  \item Interface modulaire : privilégier des alertes visuelles/sensorielles simples (LED, vibration, son), modulables par l’utilisateur,
  \item Systèmes d’analyse distribués : partager les calculs entre un smartphone, un capteur embarqué et le cloud pour alléger la charge locale.
\end{itemize}
\vspace{0.5cm}

Pour conclure, la technologie LIDAR peut représenter à elle seule entre 12\% et 84\% du prix d'un deux-roues neuf mais le pourcentage reste moindre pour des voitures haut de gamme avoisinant les 50 000 euros. Cela reste très élevé pour un motard, surtout si l'on considère que la majorité des motards ne changent pas de moto tous les ans. De plus, il faut prendre en compte le coût de l'assurance, de l'entretien, du carburant, etc. Il est donc essentiel de trouver un équilibre entre sécurité et coût pour rendre ces technologies accessibles à tous les motards.

%Autre IOT
Concernant les autres technologies que peuvent proposer les autres marques, elles sont très pertinentes car elles sont destinées à la sécurité des deux-roues et parfois élaborées par des motards eux-mêmes. Je pense à Géoride par exemple, qui est une application complète et qui répond parfaitement aux besoins pour des prix compétitifs.\\
Un enjeu particulièrement intéressant réside dans l’interopérabilité entre les différents systèmes IoT. Par exemple, des dispositifs comme Géoride qui propose un système de détection de chute pourraient être connectés à d’autres équipements comme un intercom Cardo, ou encore directement à la moto si elle dispose d’un système intégré similaire. Une telle synchronisation permettrait d’éviter la redondance comme le déclenchement de plusieurs appels d’urgence pour un même incident.

\subsection{ Proposition d’adaptation technologique}
\commentaire{	\\
•	Systèmes de communication V2X (Vehicle-to-Everything) pour motos : miniaturisation, portabilité, est ce que cela est possible ? les conditions.\\
	•	Capteurs embarqués sur la moto et sur le pilote (gilet, casque, smartphone).\\
	•	Intégration d’IA pour la détection du risque en temps réel : freinage d’urgence, angle d’inclinaison, anticipation de collision. => a voir\\
	•	Systèmes d’alerte connectés avec autres usagers (voitures, infrastructures).\\
	•	Possibilité d’un écosystème IoT dédié aux motos, interopérable avec celui des voitures. => à voir si ca existe déja\\
  • modélisation d'un tableau de bord connecté pour moto, avec des capteurs et des alertes en temps réel.\\
}

%PROPOSITION 1 : GPS MOTO 
\todo{parler également de l'accélération}

Pour pouvoir proposer de nouvelles fonctionnalités de prévention, il faut optimiser le tableau de bord. Il faut créer des logos communs à tous pour avoir le même langage. 

\todo{ajout d'un tableau de bord connecté avec des capteurs et des alertes en temps réel.}
L'idée serait d'intégrer dans les motos une puce GPS qui serait capable d'avertir en cas de virage dangeureux. L'alerte (sonore via un intercom et voyant au tableau de bord) serait envoyée via le tableau de bord connecté si la vitesse est supérieure à 5 km/h à la vitesse limite autorisée. Généralement, les virages sont signalés par des panneaux et par une limitation de vitesse. Des chercheurs Alex Liniger et Simon Hecker ont développé un prototype , Aegis Rider AG\cite{vitesse_virage_mcnews} permettant de prendre la meilleure trajectoire. Cependant, ce dernier ne prend pas en compte les autres facteurs de la route (autres usagers, état de la chaussée, etc.). 

\begin{figure}[h]
    \centering
    \includegraphics[width=0.7\textwidth]{coeur_memoire/images/aegis.png} 
    \caption{Prototype Aegis Rider AG pour la détection de virages dangereux.}
\end{figure}
Cette fonctionnalité est très interessante mais elle empêche une bonne visibilité de la surface de la route et elle peut fausser une prise de décision.
Comme illustré dans la Figure~\ref{fig:trajectoire_securite_difficulte}, le processus ne pourra pas adapter sur des virages dit "imparfaits".









\todo{voir pour faire une petite ligne de code ?}

\underline{Contexte:} En balade dans le 77, un motard roule sur une départementale. Dans cette situation, il est équipé d'un boitié GPS qui récupère ses coordonnées en temps réel. Il arrive dans une portion de virages limitée à 70 km/h nommée "Les 17 virages", près d'Arbonne-La-Forêt. Cette série de virages est dangeureuse car la route n'est pas bonne, n'est pas large et l'adhérence n'est pas optimale. De plus, il y a un virage dangereux à l'équerre qui surgit au milieu de cette série. La trajectoire de sécurité est fortement recommandé. Par expérience, avoisiner les 70 km/h est déjà bien au vue de la portion qui y est technique. 

\begin{figure}[H]
    \centering
    \includegraphics[width=0.7\textwidth]{coeur_memoire/schéma/Capture d’écran 2025-07-24 à 15.45.48.png} 
    \caption{Point GPS des 17 virages.}
\end{figure}

\todo{photo des 17 virages}



Les coordnnées GPS sont : 48.385171, 2.563108.

\todo{mettre le diagramme}
Voici le diagramme d'action de cette fonctionnalité:\\

\begin{figure}[H]
    \centering
    \includegraphics[width=0.8\textwidth]{coeur_memoire/schéma/Capture d’écran 2025-07-24 à 17.48.10.png} 
    \caption{Diagramme d'action du Système de prévention de virages dangereux}
\end{figure}

Pour réaliser un bout de code sur cette fonctionnalité, j'ai décidé d'utiliser ces bibliothèques :\\
• osmnx\cite{osm_doc} : permet d’interroger OSM (OpenStreetMap) et de récupérer des graphes routiers.\\
•	geodesic (de geopy \cite{geopy}) : mesure la distance réelle (en mètres) entre 2 points GPS.\\

L'interêt de calculer la courbure de la courbe est de pouvoir anticiper le virage et par conséquent, adapter la vitesse pour optimiser l'adhérence, la trajectoire où l'on se sentira le plus en sécurité. \\
Le calcul de la courbure permettra d'identifier un virage s'il est dangereux à partir de données GPS cartographiques pour enfin adapter le comportement du système embarqué (alerte, adaptation de trajectoire, assistance..).
Donc la courbure mesure à quel point une route peut changer de direction sur une courte distance, ici, dans un virage.\\
Une route droite a une courbure environ égale à 0. Une route qui tourne fort (virage serré) a une courbure élevée.

\begin{table}[ht]
\centering
\begin{tabular}{|l|c|c|c|}
\hline
Route & Rayon du virage & Courbure (simplifiée) & Risque \\
\hline
ligne droite & infini & 0 & faible \\
Virage large autoroute & 500 m & Faible (0.01) & faible \\
Virage serré en montagne & 30 m & Forte (0.1–0.2) & Élevé \\
\hline
\end{tabular}
\caption{Exemple en pratique}
\end{table}

Un virage serré avec une courbure > 0.1 est souvent dangereux à +60 km/h, surtout à moto.
L'avantage d'avoir un calcul automatique et en amont, cela permet de prévenir avant même d'être dans le virage. Il peut remplacer un panneau quand celui-ci n'est pas visible.



\begin{tcolorbox}[title=Calcul de la courbure]
a, b et c sont trois points dans la courbe.\\
a + b = distance réelle parcourue en suivant la route \\
c = distance directe entre le début et la fin (comme si on traçait une corde)
\[
curvature = abs((a + b - c) / (a + b))
\]
\end{tcolorbox}

\begin{figure}[H]
    \centering
    \includegraphics[width=0.7\textwidth]{coeur_memoire/schéma/Capture d’écran 2025-07-22 à 16.27.35.png} 
    \caption{Schéma présentant une moto avant un virage}
\end{figure}


\todo{illustration, développement}

\todo{Détailler le code}

\lstinputlisting[language=Python]{coeur_memoire/programme1.py}

\todo{Déscription de l'exmple}

\todo{Mettre le budget pour développer la faisabilité}

\newpage
\subsection{ Étude de faisabilité et limites}
\commentaire{\\
    •	Quels obstacles (coût, poids, énergie, connectivité, acceptabilité des motards) à l’implémentation ?\\
	•	Quelles pistes pour la recherche ou le développement industriel ?\\
	•	(une maquette fonctionnelle, un prototype conceptuel, ou même une étude de cas simulée)\\
  •	ouverture défi environnemental\\
  •	 protection des données sensibles (lien via kappa)        }


\subsubsection{Objectif de l'étude}
Cette étude vise à évaluer la faisabilité de la mise en œuvre d’un système d’assistance à la conduite basé sur l’analyse de la courbure de la route, dans le but de recommander une vitesse adaptée. Le système exploite des données géographiques pour déterminer la géométrie des segments routiers en fonction de la position d’un véhicule. Puis, il applique un modèle simple de calcul de courbure pour estimer la dangerosité d’un virage et proposer une vitesse conseillée. Ce chapitre examine les obstacles techniques, les contraintes d’implémentation réelle, ainsi que les perspectives de développement.


\subsubsection{Faisabilité technique}
Mon programme utilise des données routières open source via OpenStreetMap (OSM). Ces données sont accessibles et gratuites cependant, on apperçoit un manque de précision surtout en milieu rural. Certains segments peuvent manquer de points ou contenir des simplifications qui faussent l’évaluation de la courbure.
La méthode mise en œuvre repose sur le calcul géodésique entre trois points consécutifs, p1, p2 et p3 sur un segment de route à partir d'un point GPS, la position actuelle. Cela permet d’obtenir une estimation simple de la courbure.
Cependant, cette approche reste sensible à la densité des points sur les segments (peu de points donc cela implique une mauvaise précision).

\subsubsection{Faisabilité d’implémentation sur un véhicule réel}
\todo{reformuler}
Le monde du deux-roues posent certaines contraintes matérielles et logicielles. En effet, les capteurs nécessite d’un GPS de très bonne précision, d'un module RTK pour éviter les erreurs de localisation afin que les calculs soient optimisés. La connectivité doit être parfaite si les cartes ne sont pas embarquées, cela peut poser des limites en zone blanche. Une connexion réseau sera nécessaire. Concernant le matériel embarqué, nous pouvons utiliser un  microcontrôleur\footnote{ (ex : Raspberry Pi, Arduino, ESP32) capable d’exécuter les traitements de calcul ou de les transmettre à une plateforme distante.}. La consommation de ces appareils ne doivent pas être trop importante car ça reste un "petit véhicule" avec une petite batterie. Cela pourrait impliquer des pannes comme un alternateur ou une bobine HS.
Les limites actuellement restent les prix. En effet, ajouter des fonctionnalités de sécurité ayant des prix trop important baissent l'attractivité des motos. 

Comme il n’existe pas encore de produit commercial alliant GPS et l'alerte virage, on peut estimer sur la base des coûts de prototypes et de composants.\\

\begin{table}[h!]
\centering
\begin{tabular}{|p{7cm}|c|}
\hline
\textbf{Élément} & \textbf{Estimation de coût} \\
\hline
Capteur GPS + IMU (accéléromètre / magnétomètre) & 50–200 € \\
Abonnement cartes HD (HERE, TomTom, etc.) & 10–30 € / mois \\
Interface casque ou écran moto (HUD / retour haptique) & 100–300 € \\
\hline
\end{tabular}
\caption{Estimation des coûts des principaux composants d'un système d’assistance moto}
\label{tab:couts-composants}
\end{table}


\vspace{0.5cm}
Afin d'avoir un GPS précis et réactif. Il doit être accompané de la technologie IMU. IMU permet de compenser la latence et les imprécisions GPS en fournissant la vitesse angulaire (gyroscope) pour la courbure des virages, l'orientation (magnétomètre) et l'accélération linéaire (accéléromètre) pour les freinage. L'IMU peut fonctionner à 100-1000 Hz ce qui permet d'améliorer fortement le temps de réaction. Ces points sont crutiaux pour une fonctionnalité comme la notre, la prédiction d'abord de virages dangereux. La marge d'erreur n'est pas permise.

\vspace{0.5cm}
Voici un tableau comparatif:\\
\begin{table}[h!]
\centering
\begin{tabular}{|p{4.5cm}|c|c|}
\hline
\textbf{Propriété} & \textbf{GPS} & \textbf{GPS + IMU} \\
\hline
Précision de position & Environ 3 à 5 m & Bonne \\
Temps de réaction & Lent (de 0,5 à 1 seconde) & Rapide (<0,1 seconde) \\
Fiabilité en virage & Faible & Haute \\
Sensibilité au signal & Oui & Moins critique \\
\hline
\end{tabular}
\caption{Comparaison des performances entre un GPS seul et un système GPS couplé à une centrale inertielle (IMU)}
\label{tab:gps-vs-imu}
\end{table}


\vspace{0.5cm}
L’acceptabilité d’un tel système dépend également du profil du conducteur. Un motard préfèrera un système non intrusif (affichage sur smartphone, retour haptique) et intuitif. Le système doit être discret afin de ne pas distraire l’attention ni le surcharger d’information. Un bip, un logo pourraient constituer une solution ergonomique.


\subsubsection{Prototype et simulation}
Le prototype logiciel développé offre plusieurs fonctionnalités clés :
le chargement dynamique d’un réseau routier à partir d’un point GPS,
l’identification du segment de route le plus proche,
le calcul de la courbure de ce segment,
la recommandation d’une vitesse adaptée à cette courbure,
ainsi que la visualisation de l’environnement routier et de la position actuelle du véhicule.
Ce prototype constitue une première base fonctionnelle, ouvrant la voie à de futures expérimentations sur le terrain.

Les évolutions envisagées pour les prochaines étapes du projet sont les suivantes :
l’intégration du système dans une plateforme embarquée,
l’ajout d’un suivi GPS en temps réel,
le développement d’une interface utilisateur, visuelle ou auditive,
et la réalisation de simulations ou d’expérimentations sur circuit fermé.

\subsubsection{Limites et contraintes}
Lors de l’évaluation de la dangerosité d’un virage, la courbure n’est pas le seul facteur à prendre en compte. D’autres éléments jouent un rôle déterminant, tels que l’état de la chaussée, les conditions météorologiques, la visibilité, la signalisation ou encore la présence éventuelle d’obstacles. Ces paramètres peuvent considérablement modifier le niveau de risque perçu ou réel, même pour un virage à géométrie simple.
Le système repose sur un algorithme volontairement simplifié afin de garantir une certaine robustesse. Cette approche permet une exécution rapide et stable dans divers contextes, mais elle peut entraîner un manque de précision, notamment lorsque les conditions réelles diffèrent des hypothèses de calcul. Par ailleurs, la fiabilité globale du système dépend fortement de deux facteurs externes : la fréquence de mise à jour de la cartographie utilisée et la précision du positionnement GPS. Une carte obsolète ou un signal GPS perturbé peut nuire à la qualité des recommandations fournies.
Dans un contexte d’usage réel, le comportement du conducteur reste une variable essentielle. Il est possible que ce dernier ignore tout simplement les recommandations émises par le système, soit par choix, soit par inattention, soit par manque de confiance. Cela limite l’impact potentiel du dispositif sur la sécurité.
Il est également crucial de prévenir toute dépendance excessive du conducteur à ce type de système d’assistance. Une automatisation trop poussée pourrait entraîner un relâchement de la vigilance ou une perte de capacité d’analyse en cas de situation imprévue. Enfin, en cas de dysfonctionnement du système ou de latence dans les calculs, l’information transmise pourrait être erronée ou inadaptée à la situation. Ce type de défaut présente un risque non négligeable, d’autant plus critique si l’utilisateur se repose entièrement sur la technologie pour prendre ses décisions de conduite.

\subsubsection{Confidentialité et Données}
%1. Cadre juridique international autour du GPS
Le développement de technologies d’assistance à la conduite basées sur la géolocalisation, comme celle proposée dans ce projet, soulève d’importants enjeux juridiques, notamment en matière de souveraineté, d’accessibilité aux données cartographiques, et d’usage du GPS selon les pays.
Dans certains États, l’accès aux données cartographiques est limité. Par exemple, la Chine impose l’utilisation de fournisseurs cartographiques agréés par l’État, avec des restrictions sur les données affichées ou exportables. En Autriche et en Allemagne, les dispositifs de géolocalisation sont autorisés, mais l’usage de détecteurs ou d'avertisseurs de radars est strictement interdit, y compris lorsqu'ils sont intégrés à un GPS. Le Japon autorise l’usage du GPS, mais certains dispositifs RF\footnote{Radio Fréquence : Ce sont des dispositifs qui émettent et recçoivent des ondes radio pour communiquer entre 3 kHz et 300 GHz.}, notamment ceux qui interfèrent avec d'autres systèmes, sont interdits.

Ces exemples illustrent la diversité des contraintes réglementaires à l’échelle internationale, ce qui constitue un défi pour tout projet technologique visant un déploiement global.\\

\vspace{0.5cm}
%2. Conformité avec le Code de la route français
En France, l'utilisation d’un GPS est autorisée dans un véhicule, y compris à moto, tant qu’il ne compromet ni la visibilité, ni la sécurité du conducteur. Le Code de la route – Article R412-6\cite{loi_code_de_la_route} précise que "tout conducteur doit se tenir constamment en état et en position d'exécuter commodément et sans délai toutes les manœuvres qui lui incombent".
Ainsi, tout système d’affichage (écran, casque connecté, retour haptique) devra être conçu de manière à ne pas distraire le conducteur ou gêner sa conduite. Ce point est d’autant plus crucial sur les deux-roues motorisés, où la perte d’attention ou une mauvaise visibilité peut avoir des conséquences graves.
\vspace{0.5cm}
%3. Protection des données personnelles (RGPD)
L'utilisation d'un système GPS implique nécessairement la collecte de données personnelles sensibles, notamment :
\begin{itemize}
\item La position géographique (latitude et longitude),
\item La vitesse instantanée,
\item L'horodatage précis de chaque relevé (timestamp).
\end{itemize}
Ces informations peuvent permettre d'identifier les comportements individuels ou les trajets récurrents d'un utilisateur. Par conséquent, tout dispositif embarqué ou toute application qui traite ces données doit impérativement respecter le Règlement Général sur la Protection des Données (RGPD), en vigueur au sein de l'Union européenne.\\
Afin d'approfondir et d'améliorer le fonctionnement du système, il sera nécessaire de collecter, de suivre et d'analyser ces données GPS. Cependant, avant toute collecte, il convient d'obtenir le consentement explicite et éclairé du motard. Ce consentement doit être associé à une déclaration ou une politique de confidentialité transparente, expliquant clairement la finalité de la collecte, l'utilisation prévue et les modalités de stockage des informations.

\vspace{0.5cm}
\underline{Exemple d’organisation : expérience à Kappa Santé}\\
Lors de mon alternance chez Kappa Santé, j'ai pu observer un exemple pertinent d'organisation pour le traitement rigoureux des données sensibles. Chaque collaborateur dispose d'un accès strictement limité aux données nécessaires à ses activités. Par exemple, les développeurs utilisent exclusivement des données fictives, évitant ainsi tout risque d'exposition accidentelle ou de manipulation involontaire de données confidentielles.
Les données réelles, destinées aux analyses approfondies et aux études spécifiques, sont uniquement manipulées par des collaborateurs dûment habilités. Cette organisation repose sur l'utilisation de plusieurs bases de données distinctes, dont l'accès est soigneusement régulé selon les rôles et les droits attribués à chaque utilisateur.\\
Ce type d'organisation présente plusieurs avantages majeurs, notamment la discrétion, la confidentialité accrue des informations, ainsi que la réduction significative des risques en matière de sécurité des données. Un tel modèle pourrait être envisagé comme une référence, en particulier dans d'autres contextes sensibles tels que la mobilité connectée ou la sécurité routière. En garantissant à la fois l'efficacité et la protection des informations personnelles, ce modèle renforce la confiance des utilisateurs et assure une conformité stricte aux réglementations en vigueur.\\

Principes clés à respecter :
Dans le cadre du traitement de données personnelles, voici les principes fondamentaux à appliquer :
\begin{itemize}
\item Minimisation des données : collecter uniquement les données strictement nécessaires à la réalisation des objectifs fixés.
\item Consentement éclairé : l'utilisateur (ici, le motard) doit être clairement informé de l'objectif précis de la collecte et fournir explicitement son accord.
\item Sécurité du stockage : assurer le stockage des données dans une base sécurisée, idéalement chiffrée, et limiter la durée de conservation au strict minimum requis.
\item Transparence : accompagner tout dispositif, service ou prototype d'une politique de confidentialité accessible et compréhensible par l'utilisateur.
\end{itemize}

Exemples de données pouvant être conservées conformément au RGPD :
À titre indicatif, voici des exemples précis de données pertinentes, conservées de manière conforme et sécurisée :
\begin{itemize}
\item Latitude et longitude,
\item Vitesse instantanée,
\item Horodatage précis (timestamp),
\item Identifiant utilisateur pseudonymisé (non directement identifiable).
\end{itemize}

Ces pratiques garantissent le respect de la vie privée des usagers tout en permettant d’exploiter les données efficacement dans le cadre de recherches ou d’améliorations technologiques.\\

\vspace{0.5cm}
%4. Contraintes techniques liées à la performance et à la sécurité
Outre les aspects juridiques, la spécificité des motos implique des contraintes techniques supplémentaires. En effet, les deux-roues sont capables d’accélérations rapides et atteignent des vitesses élevées. Le système embarqué doit donc :
\begin{itemize}
  \item Être capable de traiter et d’analyser les données en quasi temps réel,
  \item Offrir une précision suffisante des points GPS (idéalement < 2 m),
  \item Fournir des recommandations sans latence significative,
  \item Être résilient aux pertes de signal GPS, notamment dans les tunnels ou zones urbaines denses.
\end{itemize}
Ces exigences imposent le choix d’un matériel performant, fiable, et optimisé pour des usages embarqués dans des conditions parfois extrêmes (vibrations, chaleur, humidité).
\vspace{0.5cm}
\todo{choix, marque du produit}
Voici des propositions de composants que nous pouvons utiliser : 	\\
•	u-blox NEO-M8N : excellent rapport qualité/prix, ~40 € \\
•	u-blox ZED-F9P : haute précision RTK, mais cher (~200–250 €) \\
•	SparkFun GPS + IMU : combine ZED-F9P + BNO080 (IMU)\\

\vspace{0.5cm}
\todo{a relire +maj}
Le programme développé fonctionne correctement dans la majorité des cas. Toutefois, certaines limites subsistent. En particulier, la récupération des données relatives aux limitations de vitesse ne se fait pas toujours de manière fiable. En conséquence, une limitation par défaut de 80 km/h est appliquée lorsqu’aucune information précise n’est disponible. Sur les portions d’autoroute, la limitation de vitesse est généralement de 130 km/h, et les valeurs calculées pour la courbure sont quasiment nulles, ce qui n’entraîne pas de recommandations particulières.\\
Cependant, une situation spécifique peut se présenter sur des voies rapides limitées à 110 km/h : dans ce cas précis, la courbure du virage doit être suffisamment prononcée pour déclencher une alerte à 80 km/h. Cette décision représente un compromis assumé dans la conception du programme, fondé sur une approche prudente pour garantir la sécurité du conducteur.\\
Du point de vue des utilisateurs, la fonctionnalité proposée pourrait être perçue comme trop « prévoyante » par certains motards expérimentés. En effet, plus le motard accumule de kilomètres et d'expérience, particulièrement s'il varie régulièrement ses trajets et types de routes, plus sa capacité à anticiper naturellement les dangers s'améliore. Ainsi, pour ces usagers aguerris, l'alerte pourrait sembler déclenchée prématurément.\\
Pour remédier à cette perception, il pourrait être pertinent d’introduire une variable personnalisable définissant différents niveaux d’alerte. Cette adaptation permettrait de moduler le déclenchement en fonction d’une vitesse spécifique, ajustée selon l’expérience ou les préférences du motard. Un autre facteur pertinent à considérer serait l'accélération : une accélération soudaine ou inhabituelle avant un virage pourrait, par exemple, déclencher une alerte anticipée afin de prévenir tout danger potentiel.\\
Enfin, il est important de souligner que le programme actuel (illustré en Figure~\ref{fig:cartepoints}) se base exclusivement sur les données de l’instant présent. Il ne prend pas encore en compte les données relatives à la trajectoire prévisionnelle ou à la direction future du véhicule. Cette dimension prédictive pourrait être intégrée dans une future évolution du programme, permettant ainsi une anticipation encore plus efficace et précise des situations à risque.

%conclusion
\vspace{0.5cm}
Le développement d’un système d’assistance à la conduite basé sur la courbure routière et la géolocalisation soulève des enjeux techniques, légaux et éthiques. Il est impératif de concevoir un dispositif conforme à la réglementation locale, respectueux de la vie privée, et suffisamment robuste pour un usage en conditions réelles. Intégrer dès la phase de prototypage les dimensions de confidentialité, de sécurité des données et de respect des législations nationales constitue un levier essentiel pour la viabilité du projet, autant sur le plan juridique que sur celui de l’acceptabilité sociale.




\newpage
\subsection{Apport personnel et positionnement}
\commentaire{Une prise de conscience collective}
La sécurité n’a pas de prix. C’est un principe fondamental qui devrait guider tout développement technologique appliqué aux transports. Si les voitures bénéficient depuis plusieurs années de dispositifs de sécurité avancés comme de l’ABS à l’assistance au maintien de voie, les deux-roues motorisés restent, en comparaison, bien plus vulnérables. Et pourtant, la pratique du deux-roues s’étend, elle attire un public passionné, exigeant, conscient des risques mais aussi demandeur d’innovation.\\
Aujourd’hui, la réflexion autour de la sécurité moto ne doit pas uniquement se limiter à des équipements de protection ou à des comportements individuels. Elle doit intégrer une vision systémique, dans laquelle l’environnement, les infrastructures, la technologie embarquée et l’intelligence collective des usagers jouent un rôle complémentaire.\\
\commentaire{Une technologie encore en transition}
De nombreux projets technologiques sont en cours. L’ABS est désormais obligatoire sur les motos de plus de 125 cm³ et certaines marques comme BMW, Honda ou Yamaha explorent déjà des systèmes d’assistance avancés : 
\begin{itemize}
	\item Détection d’angle mort,
	\item Alertes de collision,
	\item Freinage adaptatif.
	\item ...
\end{itemize}
Mais ces dispositifs restent encore peu répandus, coûteux ou en phase de test. La moto, par sa nature même :
\begin{itemize}
	\item Équilibre dynamique,
	\item Surface réduite,
	\item Exposition aux éléments.
\end{itemize}
Représente un défi technique beaucoup plus complexe que la voiture. Tous les paramètres doivent être pensés avec précision comme l'inclinaison, l'adhérence, la vision périphérique, le comportement du pilote, l'état de la chaussée, etc.\\
Il est difficile de modéliser l’ensemble de ces facteurs dans un algorithme simple. Certaines situations et certains facteurs restent imprévisibles. Et parfois, aucune formule ne suffit à expliquer un comportement ou une prise de décision sur la route. Certaines actions sont sur l'instinct. D’où l’importance de ne pas tout miser sur l’automatisation mais de construire des outils d’assistance intelligents conçus comme un appui à la prise de décision, et non comme un remplacement du jugement du pilote. C'est pour cela qu'évoluer une trajectoire parfaite n'est pas facile.\\
\commentaire{Une évolution des mentalités}
Heureusement, les mentalités évoluent. Le motard d’aujourd’hui n’est plus seulement un amateur de sensations fortes, c’est aussi un usager averti, souvent bien informé, conscient des limites de sa machine et soucieux de sa sécurité. À condition qu’elles soient bien expliquées, justifiées et qu’elles apportent une réelle valeur ajoutée à l’expérience de conduite, les innovations technologiques ont de fortes chances d’être acceptées.
Il ne s’agit donc pas simplement de créer une technologie nouvelle mais de proposer des solutions pertinentes, pensées avec les utilisateurs, testées dans des conditions réelles, et intégrées dans un écosystème cohérent. La sensibilisation et l’éducation auront également un rôle clé à jouer pour accompagner cette transition. Nous pourrons nous référer aux moniteurs écoles par exemple, aux constructeurs...
De plus, des lois doivent être plus claires et mieux mises à disposition des concernés afin de ne pas être surpris.\\
\commentaire{Une vision d’ensemble : combler le fossé}
À terme, ce travail s’inscrit dans une dynamique plus large : celle de combler le fossé technologique entre la sécurité automobile et la sécurité des deux-roues motorisés. Ce n’est qu’en intégrant les spécificités de la moto dans les démarches de conception, de régulation et d’innovation que l’on pourra répondre aux besoins réels des motards. Une mobilité durable, inclusive et sûre ne pourra exister sans penser à ceux qui, chaque jour, prennent la route sans carrosserie pour les protéger.
