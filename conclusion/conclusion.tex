\section{Conclusion}


\subsection{Conclusion mémoire}
Mes recherches n’ont pas été évidentes à mener car mon domaine professionnel  n’est pas tourné vers les véhicules ou l’automatisation. Malgré cela, j’ai eu la chance de pouvoir aborder un sujet qui me tenait particulièrement à cœur, ce qui m'a inspirée et motivée tout au long de ce travail.\\
Au cours de cette démarche, j’ai pu faire des rencontres enrichissantes et assister à des travaux et des échanges passionnants. Cela m’a offert la possibilité d’aborder l’univers de la moto sous un angle scientifique en découvrant des contraintes techniques et opérationnelles qui ne sont pas toujours évidentes à percevoir en tant que simple utilisatrice.\\
Même si, à première vue, il n’y avait pas de lien direct entre ce sujet et les missions réalisées durant mon alternance, j’ai pu observer que certaines problématiques restaient similaires. C’est notamment le cas de la protection et de la gestion des données sensibles. Qu’il s’agisse de santé ou de mobilité connectée, le défi reste le même : garantir la sécurité et la confidentialité des données sensibles tout en exploitant leur potentiel technologique.\\

\vspace{0.5cm}
Développer des technologies pour améliorer la sécurité des deux-roues représente aujourd’hui un défi réel et complexe. De nombreux aspects restent encore à explorer, et les solutions existantes sont loin d’être suffisantes pour répondre à tous les besoins. Ce travail m’a permis d'entrevoir à quel point le sujet est vaste, stimulant, mais aussi combien de zones d’ombre persistent dans ce domaine encore peu exploré.


\subsection{Conclusion année d'alternance}
Cette année, ainsi que ce travail, viennent clôturer trois années très enrichissantes chez Kappa Santé. Durant ces années, j'ai acquis de nombreuses compétences techniques, notamment dans le cadre des missions propres au métier de développeur, telles que l'analyse, la programmation, la gestion rigoureuse des données sensibles et l’intégration de bonnes pratiques dans mes processus de travail.\\
Mais au-delà de ces aspects purement techniques, cette expérience m'a également beaucoup apporté sur le plan humain et relationnel. J'ai eu la chance de travailler au sein d'une équipe dynamique et pluridisciplinaire, ce qui m’a permis d’apprendre, à comprendre les besoins spécifiques d'une étude et à développer diverses qualités essentielles dans le monde professionnel.\\

\vspace{0.5cm}
Aujourd’hui, après quatre ans d'alternance, je suis prête à me diriger vers de nouveaux horizons. Ce changement constitue une opportunité pour approfondir d'autres domaines d'expertise, découvrir de nouveaux secteurs et continuer à évoluer tant professionnellement que personnellement. Je suis enthousiaste à l’idée de relever de nouveaux défis, tout en conservant précieusement les enseignements tirés de cette expérience marquante à Kappa Santé. Merci Kappa Santé.

