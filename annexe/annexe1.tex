\section{Échange avec Hichem Arioui}
Voici l'échange transcrit que j'ai eu avec M. Hichem Arioui sur les deux-roues motorisés le 9 mai 2025 à l'Université d'Évry (Bâtiment Île de France). \\
Hichem Arioui : Donc tu fais quel parcours exactement? \\
Shana : En moto, moi, je fais tout. Que ce soit de la route, l'autoroute, des chemins de campagne.\\
Hichem Arioui : D'accord. \\
Shana: Et j'aimerais me mettre aussi à l'off road. C'est un peu... Ce n'est pas vraiment du cross, mais c'est autre chose. \\
Hichem Arioui : Ok, très bien. Moi, [...]je fais de la moto. Je ne le fais pas ici en France, je le fais en Indonésie. Mais c'est plutôt des scooters. Ok. Des X-max, des T-max. Voilà, tout simplement. Mais je connais, c'est un domaine sur lequel je travaille depuis 2000... 2005, 2004, quelque chose comme ça. \\
Shana: D'accord. Oui, j'ai vu un peu ce que vous travaillez. Là, c'est en ce moment sur les motos électriques, pardon? Là, en ce moment, oui, \\
Hichem Arioui: c'est des motos électriques parce que c'est une difficulté un peu étrange, parce que la façon dont on accélère et on freine, le couple, il est instantané. \\
Shana: Oui, on le voit avec les Tesla, par exemple. \\
Hichem Arioui: Exactement. Mais ouais. Et pour freiner, accélérer, parfois, ça t'emmène vraiment vers des situations de dérapage que tu ne peux pas contrôler forcément. \\
Shana: Malgré l'ABS et autres? \\
Hichem Arioui: Ben oui. L'ABS, généralement, quand on peut le débloquer, les motards le débloquent. Ce qu'ils veulent pas-- ce qu'ils cherchent, les motards, pas les novices. Je parle des expérimentés. C'est plutôt la liberté de faire ce que je veux. Donc, tu n'es pas le TCS, le MCS, le machin. Ça, ils désactivent. Ils n'aiment pas. \\
Shana : Moi, j'ai la chance, c'est que j'ai une moto vraiment récente qui date de l'année dernière. J'ai changé entre temps. \\
Hichem Arioui: C'est quoi? \\
Shana: J'ai un GS, un F800GS. C'est une moto type trail. On peut faire du voyage aussi beaucoup. Et je le sens que des fois, les aides, elles arrivent et ça pardonne beaucoup de choses que ce soit... \\
Hichem Arioui: Tu as quoi comme aide dessus? \\
Shana: J'ai l'ABS, j'ai l'anti-wheeling. Donc, ça pardonne au niveau de l'accélération. Je sens que la moto veut se lever, enfin nous empêche de se lever. Donc, c'est vrai que pour une personne qui vient de débuter, ou que ce soit la moto, ou qui découvre la moto, ou qui est un peu brute, ça pardonne beaucoup. Et la plupart des motos ne l'ont pas forcément. En fait, j'ai pas mal de connaissances dans la moto en termes d'amis, etc. Et d'autres qui ont fait des chutes. Et en fait, avec leur retour d'expérience, c'est intéressant. En fait, à côté, je suis pompier volontaire. \\
Hichem Arioui : D'accord. \\
Shana: Donc, c'est vrai que le terme d'accident au niveau des motos, ça revient un peu plus. On se dit: OK, mais qu'est-ce qu'elle a fait la personne? Elle a trop accéléré, excèes de confiance. C'est pour ça que moi, je me dis que j'aimerais bien apporter... \\
Hichem Arioui: La moto, elle pardonne vraiment pas, c'est le terme. Oui. Moi, dans les études que j'ai faites, j'ai remarqué que c'est 25 fois à peu près plus dangereuse que le véhicule. Et les chutes, elles sont généralement fatales. Elles sont graves. C'est très peu de situations... C'est pour ça que ça m'intéresse. C'est très bien ce que tu dis, parce que généralement, quand on développe une aide, n'importe laquelle, on part sur les forums motards. \\
Shana:Pour avoir le retour d'expérience. \\
Hichem Arioui: 80\%, on n'en a rien à cirer. Ça ne nous intéresse pas. Surtout, ne faites pas ça pour la moto, parce que c'est plutôt des forums d'expérimentés. Ce n'est pas des novices. Ce qu'ils veulent, ce n'est pas d'introduire quoi que ce soit comme aide. Ils veulent être libres de toute aide. \\
Shana: C'est ça mon défi aussi, c'est de se dire comment on peut apporter plus de sécurité dans la moto sans forcément nous retirer le plaisir de conduire. \\
Hichem Arioui: Si tu trouves une réponse, je serais intéressé. Oui, je comprends, mais je me dis que... \\
Shana: En plus, j'ai des personnes qui ont par exemple le permis depuis plusieurs années, mais qui ne conduisent quasiment pas. Et la moto, pour moi, ce n'est pas comme la voiture. C'est-à-dire qu'il faut un peu s'entraîner, se dire : je la prends, je vais m'entraîner pour les virages, les trajectoires de sécurité. Parce que j'ai fait quelques stages avec les gendarmes pour avoir vraiment des notions de sécurité à ce niveau-là. 
Hichem Arioui: Lesquels exactement, les gendarmes? C'est de Fontainebleau?\\
Shana : C'est un peu tout. La gendarmerie du coup, par exemple, du Seine-et-Marne, où tu proposes des journées. \\
Hichem Arioui : C'est les plus fous, en plus. \\
Shana : Oui, c'est vrai que quand on conduit avec eux, on conduit un peu fort. \\
Hichem Arioui : J'ai fait des projets avec eux depuis une dizaine d'années maintenant. Je les connais, mais c'est... \\
Shana : Ça conduit fort. \\
Hichem Arioui: J'ai fait avec eux le... Quel circuit? Le circuit de... Peut-être Carole qui est à côté. En 77. Un circuit. \\
Shana : La Ferté-Gaucher? \\
Hichem Arioui: La Ferté-Gaucher, oui. C'est des malades. C'est des malades. Et surtout en moto électrique où c'est vraiment très super dangereux. Les mouvements qu'ils ont faits, j'ai la vidéo, je ne sais pas si je l'ai ici ou pas. Mais c'est des fous. Vraiment, c'est des fous. En fait, pour moi, ils ont fait tomber toute la théorie que j'ai faite. C'est-à-dire là où tu dis: Ce virage-là, je ne peux pas l'attaquer avec telle vitesse, ils l'attaquent avec une vitesse beaucoup plus supérieure que... Parce que nous, on prend beaucoup plus d'hypothèses théoriquement et tout. Donc, ça fait que les modèles ne sont pas complètement fidèles, mais tu le vois. Tu dis: Non, ce n'est pas possible. \\
Shana : 'ai eu l'occasion de faire aussi de la piste avec ma moto que j'ai actuellement. Donc c'est encadré par une association, toujours avec la pédagogie de la sécurité et de découvrir la moto, sa moto sur l'environnement sécurisé. Et on avait appris à déhancher, ce qu'on ne fait pas du tout sur route, mais on nous dit: On peut déhancher pour...Garder une stabilité sur la moto pour que ce soit plus droit afin que, par exemple, s'il y a des feuilles ou quoi, plus la moto est penchée, du coup, elle est vulnérable.\\
Hichem Arioui: Absolument. \\
Shana : Mais par contre, on perd en termes de champ de vision sur ce qui arrive en face. Donc là, l'idée, c'était de se dire, plus on déhanche, plus on peut passer vite et plus on peut pencher la moto pour passer vite. Et c'était très intéressant. Et après de se poser la question, c'est est-ce que si on rajoute une aide, parce que là récemment, j'étais en voiture, donc c'est une Clio 5 2019, donc qui a quelques aides. Et à un moment donné, on a des parkings, on va dire en quinconce, il faut les éviter pour ralentir. Je suis passée... Pour moi, ça passait, mais la voiture pensait que ça ne passait pas et elle a pilé d'un coup. \\
Hichem Arioui : D'accord. \\
Shana : Et je ne savais pas qu'elle pouvait faire ça. Moi, j'avais juste un voyant me disant freiner. C'est vrai que je me suis dit: mais si on reporte ça sur la moto, on ne peut pas. \\
Hichem Arioui : Le problème, c'est que tu sais... Je te donne un exemple simple, quelque chose que j'aimerais développer, on est en train de développer en ce moment. Ok. Ce que je te dis, tu le gardes pour toi parce que c'est des recherches encore... \\
Shana : Ok, pas de souci.\\
\[...\]
\ifconfidentiel
Hichem Arioui : C'est la correction de trajectoire, au sens large. C'est vraiment la correction de trajectoire parce que quand tu es sur la voiture, tu parles de ce cas-là, la voiture, c'est quatre roues, tu fais l'ESP. L'ESP, c'est quoi? C'est du freinage différentiel. C'est-à-dire quoi? C'est-à-dire que j'ai quatre roues. Si je freine celle-ci, freinage différentiel, c'est-à-dire je freine différemment sur les roues. Si je freine complètement celle-ci, la voiture, elle va faire ça. Donc en fait, et c'est comme ça que je corrige la trajectoire. C'est pour simplifier. La moto, ce sont deux roues. Si tu freines celle-ci, soit tu fais ça, ou tu fais ça, ou ça dépend de ce que tu fais exactement. Il faut que tu te penches un petit peu pour pouvoir faire... C'est extrêmement difficile de faire du freinage différentiel pour corriger la trajectoire en moto. Ce qu'on est en train de faire, ce que je suis en train de faire, c'est le regard. Ok. \\
\fi
Shana :Le regard. C'est vrai que c'est le regard qui change tout dans les virages. \\
Hichem Arioui : Il est fondamental.
Tout le monde le dit. Pour les sorties de virage, c'est le regard ou la vitesse,... \\
Hichem Arioui: Ce que j'ai fait avec un doctorant, ce qui est un travail exceptionnel pour moi euh. Je, je, je vante rarement mes travaux, mais celui-ci est exceptionnel dans le sens où, par exemple, parce que tu as parlé à juste titre du virage, où est-ce que je regarde pour faire le virage, certainement, tu as fait avec les motards le EDSR. \\
Shana : Oui. L'entrée, la découverte, exact. \\
Hichem Arioui : Exactement, la sortie de la sollicitation, machin. Exactement. Où est-ce que tu regardes exactement? Ils vont te dire: Tu regardes à l'apex ou le point de corde, voilà exactement l'apex en anglais. Et quand est-ce que tu regardes? Après, dès que tu obtiens la ligne droite pour sortie, tu commences à te redresser. Ce genre de choses. Et nous, ce qu'on a fait, ce point-là, on a fait ça en simulation virtuelle. On ne peut pas s'autoriser ça. Ouais. C'était en simulateur. Donc, ce qu'on fait, c'est que j obstrue toute la zone et je montre que l'apex. Ok. Je dis au conducteur: Je suis en train de te montrer l'apex, mais en fait, ce n'est pas l'apex, c'est un autre point. C'est juste pour tester la sensibilité de ce point-là. Donc en fait, l'apex, il est réellement ici, mais moi, je mets ici ou je le mets ici ou je le mets un peu légèrement ici à droite et à gauche. Il est fondamental dans le sens où la trajectoire est complètement différente. C'est-à-dire à 10, 15 centimètres, la trajectoire est complètement différente. Et le changement est complètement différent. La sensibilité, elle est énorme. C'est-à-dire, je pense, on a fait quoi? Comme euh comme... cinquante centimètres de mouvement. C'est-à-dire le point, il est là. Oui, donc la trajectoire est complètement différente. D'accord. Et elle est encore plus, comment dire, elle est encore plus différente si la vitesse est supérieure. Et donc, et ça dépend où est-ce que tu commences ton virage. Est-ce que tu le commences complètement collé à droite et c'est un virage à droite ou plutôt au milieu, ça dépend. Et là, tu vois que c'est énorme et c'est ce qu'on souhaite faire. Donc ça, c'est juste pour te dire que je ne vais pas te convaincre, le regard, il est capital. Ce qu'on souhaite faire maintenant, c'est avec des oculomètres. Ce que je suis en train de dire, je dis: Écoute, tu es en train de regarder là, surtout pour les novices. Ce que tu es en train de regarder là, ne regarde pas là, regarde-là. Et je sais que s'il continue à regarder ici, la trajectoire, comment elle va être et si elle est rattrapable à partir d'un certain moment ou pas et comment je peux la corrigée. Et après, il faut prendre en- Et lui dire: Ne regarde plus ici, regarde ici. Et je sais que s'il regarde là, la trajectoire, bien sûr, c'est une correction, elle n'est pas active, elle est passive. En fait, j'agis sur-- j'agis pas sur la moto, j'agis plutôt sur le regard. Et en changeant de regard, je sais comment la trajectoire va être corrigée. 
Et quand vous dites le regard, ça prend en compte aussi, par exemple, la surface de la route, s'il n'y a pas de gravier, s'il n'y a pas de... 
Mais bien sûr. En fait, je parle dans des situations où il n'y a pas de tiers, il n'y a pas de personnes sur machin. Et les routes euh, pour que je puisse calculer ce genre de choses, je connais exactement les délimitations ou le cas le plus simple, je connais le marquage au sol. J'arrive à le détecter facilement. Je ne suis pas encore sur piste, je ne suis pas encore sure. \\
Shana :Après, ça se voit façon dans un virage, il y a énormément de possibilités. Juste par exemple les gens qui font par exemple le moto GP ou les courses sur moto. Et c'est vrai que ça modifie énormément. Il y a énormément de possibilités en fait.\\
Hichem Arioui : C'est-à-dire, on se conçoit de faire dans un premier temps. Malheureusement, la recherche, c'est comme ça. Je ne propose pas tout de suite l'aide, mais ce que disent tout le temps les policiers ou les gendarmes qui sont super intéressés par cette idée-là, c'est qu'on a réussi à prouver la causalité entre le regard et la trajectoire. Parce qu'on le dit, mais on ne l'a jamais prouvé réellement. Nous, on le prouve. Ça n'a jamais été prouvé. Mathématiquement, je parle scientifiquement. Tout le monde le dit. Oui, le regard est super important, surtout les gendarmes. Il faut que tu regardes ici. Mais on l'a prouvé. Avec la sensibilité dont je te parlais tout à l'heure, c'est quelque chose de capital. Et qu'est-ce qu'on en fait de cette erreur-là, de ce biais-là? S'il doit regarder là, mais il regarde là, et la trajectoire de sécurité ou la trajectoire optimale, elle est comme ça, mais lui, il est loin. Comment je le fais converger vers cette ligne de sécurité? Juste en changeant son regard. C'est ça ce qu'on est en train de faire en ce moment. Pour moi, c'est excellent. Je ne sais pas si c'est quelque chose qui... Mais moi, je serais intéressé par le retour des situations dans le jeu. Par le retour des situations danger que tu... Auxquelles tu te pointes. Voilà. \\
Shana: j'en ai plusieurs. Avec ou sans aide. Bien sûr. J'ai pas mal de... Je pense qu'il y a déjà l'anticipation sur, par exemple, les autres. 
C'est capital.\\
Hichem Arioui : Par exemple... Ça dépend pas que toi, en fait, ça dépend des autres aussi, ce qui te voit ou pas. Ça dépend des autres.\\
Shana : Et des fois, moi, je me dis des fois: Cette voiture-là, je sens qu'elle ne m'a pas vu, je le sens. Et des fois, c'est vrai. Donc bah, je ralentis, je suis là. \\
Hichem Arioui : Ce qu'on est en train de faire, c'est avec un consortium européen qui s'appelle le CMC, Connected Motorcycle Consortium. C'est un consortium qui travaille sur les motos connectées et donc en fait, il donne une visibilité sur les motos que tu vois pas. Même dans un carrefour où elle est derrière un bâtiment, je pourrais la voir. Ou tout simplement dans un carrefour où tu veux, toi, tu es complètement à droite, tu veux prendre à droite. Elle, elle te voit pas, donc elle te crasse facilement. Des situations comme ça. \\
Shana : Je suis intervenue sur une intervention où c'est la dame qui a pas vu la moto. Ouais. Elle a dû regarder vraiment sur un rond-point à gauche. Pas de voiture, elle allait tout droit et il y avait la moto tout droit. La moto, elle était un peu en dessous, donc la personne va bien. Mais c'est vrai que ça fait une situation de se dire peut-être que le motard allait trop vite, je pense, sur la manière il l'a dit qu'il avait mis de l'angle et il avait pas regardé la voiture qu'il avait pas vue. Donc ça fait une situation en plus. Bon l'interfile ou... \\
Hichem Arioui : Bien sûr. \\
Shana : Bon après, il y a deux cas. Il y a les personnes qui roulent un peu vite et t'as le cas où la personne roule lentement, mais la voiture met pas de clignotant. Il y a cette partie-là. \\
Hichem Arioui : L'interfilement est interdit aux États-Unis. D'accord. Ok. Ici, il est interdit, mais il est toléré.\\
Shana : Maintenant, il est passé avec la FFMC. \\
Hichem Arioui : C'était quand ça? \\
Shana : C'était il y a quelques mois. C'est passé là, je pense peut-être février, quelque chose comme ça. \\
Hichem Arioui : C'est-à-dire? Ils le tolèrent ou ils l'autorisent? \\
Shana : Maintenant, c'est autorisé. Mais max 50 km/h, 20 km/h d'écart avec les voitures. Il faut pas que ce soit... Max 50, donc tu le fais pas en autoroute. \\
Hichem Arioui : Moi, je parle de l'autoroute. \\
Shana :Si la circulation dépasse pas les 50. Oui. Mais si ça roule à 40, normalement, on peut pas parce qu'il faut 20... Si, on peut. Mais on ne peut pas plus de 20 km/h. \\
Hichem Arioui: Tu peux m'envoyer, s'il te plaît, le document après?\\
Shana : Oui, je pourrais essayer de vous retrouver ça. Ok. Il y a ça. Après, il y a les personnes qui glissent parce qu'on n'a pas vu la plaque d'huile ou d'essence. Ça, ça arrive. Le bas-côté aussi. Les gravillons, les choses comme ça. \\
Hichem Arioui : Ça, c'est fatal. \\
Shana : Et aussi, des fois, avec la pluie dans un virage, des fois, moi, ça m'est déjà arrivé, la moto, elle a fait comme un petit saut. Et je pense que les roues devaient pas être alignées avec la pluie. Je sais pas vraiment ce qui s'est passé, mais avec... Posé, je me suis dit: À mon avis, ça va pas être ça. Ça, en arrivant sur Melun, 4 voix, 90 km/h et ça fait un petit virage léger, mais on peut bien le prendre en temps sec. Là, il pleuvait, mais j'avais roulé toute la journée et mes pneus adhéraient pas trop mal, donc je me suis dit : Je la prends. \\
Hichem Arioui : Il n'y avait pas de flaque d'eau? \\
Shana : C'était un peu humide. \\
Hichem Arioui : D'accord. \\
Shana : Et je pense que mes roues, elles étaient un peu de travers. Elles n'étaient pas vraiment alignées quand j'ai pris mon virage. Et je pense que ça a dû faire un... Là, que ce n'était pas... Je pense que ça a dû faire... Hop. Je me suis retrouvée sur le réservoir. Je me suis vue tomber, mais je ne l'ai pas fait tomber. Voilà, là, c'était une bonne frayeur. 
Parce que... \\
Hichem Arioui : Je le dis, mais ce n'est pas à reproduire, mais généralement, dans des situations de chute, freiner, c'est tout à fait le... \\
Shana : Il ne faut pas. \\
Hichem Arioui : Il ne faut pas. C'est exactement le contraire qu'il faut faire, généralement, pour soulever la... \\
Shana : Comme quand on guidonne. Guidage, il faut limite, ils disent de mettre un petit coup d'accélération. \\
Hichem Arioui : Exactement. Le guidonnage est à faible vitesse et à fréquence un peu particulière. \\
Shana : Ça m'est arrivé quand j'ai fait le petit saut avec la moto, je me suis retrouvée sur le réservoir, ça guidonnait un peu, j'ai laissé faire. Après, je me suis dit : C'est bon, je... Ça, ça a été une belle frayeur. Une fois aussi, sur le regard, je suis arrivée un peu vite sur un virage, j'ai voulu freiner. Là, l'ABS et tout s'est déclenchée. J'ai eu un petit problème aussi sur mon sélecteur de vitesse. Je n'ai pas pu bien faire mon freinage, mon rétrogradage et du coup, frein moteur. J'ai regardé un peu tout droit. J'ai réussi à piler, je l'ai arrêté vraiment avant, mais ça a été le regard. J'aurais pu tout lâcher, me dire... Ça m'est déjà arrivé. Donc tout lâcher, les commandes et on penche la moto, on met le regard. \\
Hichem Arioui : C'est exactement la différence entre l'expert et le novice. Face à n'importe quelle situation, le novice, il freine. L'expert, il évite. Il freine après. \\
Shana : C'est ce que je dis un peu tout le monde avec un retour d'expérience, qui viennent de commencer ou quoi. Il vaut mieux tout lâcher, pencher, mettre le regard. Il y a plus de chances de ne pas tomber en faisant ça que en freinant et en descendant tout droit. \\
Hichem Arioui : Exactement. Le novice, tout de suite, il... \\
Shana : Oui. Mais même, par exemple, il vaut mieux faire un évitement d'urgence qu'un freinage. \\
Hichem Arioui : Exactement. \\
Shana : Parce que des fois, le temps de freinage, il peut être vraiment long. \\
Hichem Arioui : Absolument. C'est super intéressant. Moi, je sais pas comment on peut revoir ça. \\
\[...\]
\ifconfidentiel
Parce que du coup, dans mon parcours, on n'est pas trop axé sur ce travail-là. Je sais, je sais. Mon entreprise, c'est pareil, mais ils m'ont accordé, on va dire, le fait de faire ce travail de recherche dessus et je trouve ça perturbant parce que j'aimerais... C'est ce que je me suis dit : Si tu es en moto, je roulais et je me suis dit : J'ai envie de faire mon devoir là-dessus. 
On a justement un...Une réunion annuelle d'un projet sur les véhicules électriques. Ok en présence de la gendarmerie mais c'est un onte ok je vais regarder si le zoom est toléré je vais t'inviter c'est le 30 juin Ok parfait quand tu m'envoies le lien ok de la nouvelle autorisation de faux film ok dis-moi juste ça comme ça je l'ai en mail moi je fonctionne comme ça ok par mail et je vais essayer de demander à l'organisateur s'il y a possibilité et te laisser avoir accès aux discussions ok voilà ça va te donner une idée un peu de ce qu'on est en train de faire et surtout en ce moment qu'est-ce qu'on fait exactement ok justement avec les gendarmes de Fontainebleau ceux qui font la préparation des films du défilé de 15 juillet ok et justement on fait pas mal de d'expérimentation avec la fierté gaucher. D'accord, ouais. Voilà. Je vois. 
\fi
Hichem Arioui : Parfait, nickel. Ok. Très bien, merci beaucoup. \\
Shana : Merci à vous.\\
Hichem Arioui : c'est super intéressant donc c'est moi qui vais pas te lâcher. \\
\[...\]
\ifconfidentiel
Bah franchement ça serait parfait parce que pour mes recherches c'est vrai que je regarde un peu sur internet je regarde il y a des recherches qui ont été faites mais très peu aux motos malheureusement moi tu sais j'ai mon doctorant Pierre-Marie Damand qui est un excellent personne il commence à travailler maintenant sur les adas oui pour les motos avec Valeo il est chez Valeo qui était la meilleure thèse en france sur la sécurité routière en deux mille dix-huit l'envoyer une année ou un mighty il a fait une partie de sa thèse ou un mighty et c'est un motard, c'est toujours un motard. D'accord. Je pense que quelqu'un qui peut être super intéressé intéressant par le par ce qu'on est en train de dire ici je vais lui demander si on peut se voir un jour Je pense que sa connaissance peut être très intéressante.
\fi

\section{Article 1}
\begin{figure}[H]
    \centering
    \includegraphics[width=1\textwidth]{annexe/images/Capture d’écran 2025-08-07 à 08.43.20.png} 
    \label{articlemoto}
\end{figure}
\begin{figure}[H]
    \centering
    \includegraphics[width=1\textwidth]{annexe/images/Capture d’écran 2025-08-07 à 08.43.41.png} 
\end{figure}


\section{Article 2}
\begin{figure}[H]
    \centering
    \includegraphics[width=0.5\textwidth]{annexe/images/Capture d’écran 2025-08-11 à 15.42.06.png} 
    \label{articlevoiture}
\end{figure}
\begin{figure}[H]
    \centering
    \includegraphics[width=0.5\textwidth]{annexe/images/Capture d’écran 2025-08-11 à 15.42.23.png} 
\end{figure}
\begin{figure}[H]
    \centering
    \includegraphics[width=0.5\textwidth]{annexe/images/Capture d’écran 2025-08-11 à 15.42.32.png} 
\end{figure}

